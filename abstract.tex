The human genome project, which began in 1988 and completed in 2001, ushered in
a new era of biology.  This effort accelerated the development of biochemical
assay and information technologies.  As a result, biologists are now able to
ask questions that were previously considered intractable.  One example of a
breakthrough in assay technology is the DNA microarray, a high-throughput
measurement device which enables individual scientists to rapidly and
simultaneously interrogate the RNA concentration levels of virtually all genes
in the human genome for a single biological source.  As with all advances, the
advent of DNA microarray has created a new frontier of challenges.  In this
document, I describe an approach that addresses the problem of assembly,
processing, and subsequent analysis of large volumes of data collected with DNA
microarray.  My work is presented in 5 chapters and 3 appendices.

Chapter \introchapter serves as a general introduction DNA microarray assay
technology, idiosyncrasies of using this technology in biological experiments,
methods for pre-processing the resulting experimental data, and techniques used
in the informatic systems that enable the processing, representation, storage,
and subsequent retrieval of these data.

Chapter \celsiuschapter is the core of the dissertation and describes the
Celsius project, a microarray data warehousing system that is an implemented
solution to the informatic problems described in \introchapter.  The completion
of the Celsius project brought into existence the single largest publicly
available source of primary and uniformly pre-processed DNA microarray data.

Chapter \corchapter builds upon Chapter \celsiuschapter by describing an
analysis of the data present in Celsius.  Specifically, it describes the
creation of gene-gene correlation matrices and their application in performing
gene annotation and identifying disease genes within known linkage regions.
While the idea of using gene-gene coexpression patterns is as old as DNA
microarray technology itself, the scale of this analysis is unprecedented and
the demonstrated applicability of the correlation data to a broad set of
biological questions raises concerns about the validity of current microarray
data deposition systems which rely heavily on experimental metadata.

Chapter \biopackageschapter presents Biopackages.net, a technical subsystem of
the data warehousing system described in Chapter \celsiuschapter.
Reproducibility is a shared pillar of both scientific and data warehousing
methods.  Because Celsius is very dependent on computing systems to process the
data stored in the warehouse, it was essential to have a mechanism for making
uniform and reproducible computing environments.  This not only allows the
system to scale as the volume of data inevitably increases, but also garners
the benefits of being able to clone the system at other sites and to recover
from failures.

Chapter \gmodwebchapter and Appendix \daschapter describe efforts for data
modeling and dissemination.  As we enter the post-genome era, new assay
technologies continue to appear, and the growth in volume of existing and new
data generated from each technology continues to accelerate.  Thus, it
imperative that protocols be developed for the encoding and distribution of
these data to both individual scientists and the information systems and agents
acting on their behalf.

Appendix \chochapter and Appendix \funarichapter present analyses performed on
previous iterations of Celsius, which is described in Chapter \celsiuschapter.
These early collaborations provided a glimpse of the utility of creating a
micorarray data warehouse, without which the work described here would never
have been completed.
