% \title{Biopackages.net: Bioinformatics Libraries, Applications, and Data as Operating System Packages}
% \author{
% Allen Day\correspondingauthor$^{1,2,\ddag}$ \email{Allen Day\correspondingauthor - allenday@ucla.edu} \and
% Brian D. O'Connor$^{2,3,\ddag}$ \email{Brian D. O'Connor - boconnor@ucla.edu} \and
% Jordan Mendler$^{1}$ \email{Jordan Mendler - jmendler@ucla.edu} \and
% Jared Fox$^{1,4}$ \email{Jared Fox - jaredfox@ucla.edu} \and
% Stanley F. Nelson$^{1}$ \email{Stanley F. Nelson - snelson@ucla.edu} \and
% Lincoln D. Stein\correspondingauthor$^{2}$  \email{Lincoln D. Stein\correspondingauthor - lstein@cshl.edu}
% }

%\maketitle


%%%%%%%%%%%%%%%%%%%%%% ABSTRACT
%\begin{abstract}

\section{Abstract}

\paragraph*{Background:}
As biological science becomes more quantitative, it is increasingly dependent on large and
complex computational infrastructure.  Large-scale systems exist upon which 
analyses can be done, but efforts are often foiled by the analytical software itself,
which is frequently of pre-production quality, poorly documented, and difficult to locate, install and configure.

\paragraph*{Results:}
We have created Biopackages.net: a repository of libraries, applications, and data sets
that are commonly used in computational biology. This resource solves the problem of software installation by providing pre-configured packages for many of the data and software most commonly used in the biological sciences.  We have also created a ``build farm''
-- a software application that allows software packages in the repository to be automatically adapted to
new operating systems and hardware architectures.

\paragraph*{Conclusions:}
Using software available from the Biopackages.net repository in conjunction with a
mature cluster management product lowers the barrier to entry for addressing
large-scale problems in computational biology and bioinformatics.  It streamlines the initial setup and maintenance of complex software collections allowing scientists to focus more time on research and less time on configuration.  More information is available at \url{http://biopackages.net}.

%\end{abstract}

%%%%%%%%%%%%%%%%%%%%%% INTRODUCTION
\section{Introduction}

As biological science becomes more quantitative, it is increasingly dependent
on large and complex computational infrastructures.  Scalable and modular clustered
computer systems are now essential for analysis and hypothesis generation from
current high-throughput assay technologies.  While advances in the scientific computing
community have led to the creation of systems that use clustered computer systems to
increase productivity, e.g. the Rocks environment \cite{rocks}. These efforts seldom address another part of the
problem, namely \textit{deploying} research-grade software on these systems.  This latter problem is not trivial, and the associated
productivity loss is a major obstacle in the advancement of computational life science.

The degree to which software increases productivity can generally be evaluated
along three axes: utility, accessibility, and usability \cite{keates}.  Applications and software libraries produced in life science research environments
are of great utility.  They provide means to represent, store, retrieve, and manipulate
biological data.  However, even the best of these applications and libraries typically have low marks
for accessibility and usability.

Research code is often not readily accessible because it is of prototype
quality, with poorly documented interface, installation and configuration
mechanisms, and comes with no formal support or guarantee that it will compile,
execute, or be otherwise usable on the platforms and operating system combinations of end users.  Usability suffers from poor documentation, as well
as from inconsistent and often rough-cut user interfaces.  Ease of use has also been
lacking in the research code environment. Software installation has historically
been a painful process that involves README files, configuration settings, and
other processes requiring active human interaction.  This condition frequently
persists even within popular and widely used scientific software, likely due to a
deficiency in knowledge of and adherence to software ``best practices'' as
previously posited \cite{baxter}.

%XXX
The primary aim of the Biopackages.net project is to increase the productivity
of life science informatics environments through software packaging.  By
compiling, testing, and configuring software, we are able to greatly increase the
accessibility and usability of it.  Another benefit of packaging is that it is possible to
compile and configure complementary packages to be aware of and interact with
one another and the underlying OS.  This creates an environment
with greater utility than the sum of its parts.

%Considering the extra problems introduced by significant increases
%in scale, as in the case of administering large numbers of systems that all
%need to have similar utilities installed -- this stratified
%installation procedure simply does not scale \cite{couch}.  One of the
%strategies used to combat these stratified installations is the use of thin
%clients. Thin clients are machines on a network that retrieve data and programs
%from central repositories on the network and typically contain no local
%customizations. This technique allows system administrators to update all of
%the clients by making a software change on the single central repository.  One
%drawback to this method is that the central repository quickly becomes very
%complex, since each user will likely have special applications or
%customizations that they need to use. In this scenario, users who install
%software onto the central repository can frequently corrupt the installations
%of other users' applications \cite{couch}. This requires the system
%administrator to spend a great deal of time installing, upgrading, and
%troubleshooting software installations.  We are not convinced that the thin
%client approach is practical, in that rather than truly factoring out the
%administrative overhead, it has instead shifted the problem into a mode of
%repository maintenance.  Thin clients also have a performance price in life
%sciences applications that are frequently I/O bound and are better served by
%local storage media than media available through the local area network.

%%%%%%%%%%%%%%%%%%%%%% RESULTS
\section{Results}

The Biopackages.net software repository is available at
\url{http://www.biopackages.net/}.  At the time of writing more than 3800 software packages %XXX 3800
are available for several versions of the Fedora Core and CentOS Linux operating systems on
i386 and x86\_64 architectures, as well as a shared set of
architecture-independent (noarch) packages.  All packages currently provided
are prepared using the Biopackages.net build farm which in turn employs the RPM
Package Manager \cite{herrold}, a suite of utilities for installing and
maintaining software on a computer system.

\subsection{Provided Applications and Libraries}

Biopackages.net aims to provide a software repository as broad as possible
within the discipline of open-source bioinformatics.  Research interests of the
contributing authors that are outside the scope of this manuscript include
complex disease gene mapping and gene expression analysis.  As such, there is a
trend for the content of the repository to reflect these topic areas.  Even so,
we have attempted to include many of the commonly used bioinformatics data
sets, libraries, and applications.  Data sets provided include: ontologies
provided by the National Center for Biomedical Ontology, standard Affymetrix
GeneChip data sets, and recent genome assemblies for several commonly studied
organisms pre-formatted for use with the sequence analysis programs BLAST,
ePCR, and BLAT \cite{ncbo,blast,epcr,blat}.  For libraries, the
popular Bioconductor and Bioperl \cite{bioconductor,bioperl} libraries are
available, as well as the NCBI Toolkit libraries supporting the BLAST
application already mentioned.  Other applications available include the
Generic Genome Browser, the BLAT server, Textpresso, EMBOSS, and Turnkey/GMODWeb
\cite{gbrowse,blat,textpresso,emboss,gmodweb}.

\subsection{Using the Biopackages.net repository}

Software provided by the Biopackages.net servers can be browsed and downloaded
directly using a web browser, then installed on the target system with the
\texttt{rpm} command-line utility.  Using the RPM Package Manager
reduces the amount of human interaction to two steps in most cases: users can
simply retrieve the RPM file and then install with a single command.  In
addition to ease of installation, RPM also provides a wide array of features
that make it superior to many other software packaging formats \cite{hess}.

While this method of installing software is possible we also support -- and
recommend -- using automated software utilities for package dependency
resolution and installation, such the Yellow dog Update Manager (Modified), or
\texttt{yum} \cite{vidal}.  \texttt{yum} allows users to install desired packages in an automated
fashion with a single command.  The Biopackages.net website provides
instructions and a single RPM that can be downloaded and installed to make the
user's installed version of \texttt{yum} configured to be aware of how to obtain and install packages from the Biopackages.net repository.  \texttt{yum} is well
suited for bioinformatics research because many of our applications and
libraries are considered high-level from the point of view of the
OS and have a deep and broad graph of software dependencies.  For example,
consider the partial software dependency graph shown in Figure
~\ref{fig:bp_depgraph}.  Installation of the \texttt{das2-server} package
requires the \texttt{chado-Hsa} package, which in turn requires two other
packages to be installed, and so on.  Installation of these high-level packages
frequently requires the manual retrieval and installation of possibly hundreds
of other packages in the dependency graph \cite{bodnar}.

%\subsection*{Performance}

%Many aspects of system performance, such as network I/O and disk usage can be
%affected by using Biopackages.net repository, with the gains or losses
%incurred through use of this repository depending on the machine and network
%configuration at a particular site.  As this is the case, we have chosen to
%evaluate performance gains of using the Biopackages.net repository in terms of
%human administrative costs, which should be consistent between sites.
%\textbf{Bar chart of build+install vs install time here}

\subsection{Build Farm}

Software packages for the biopackages project are built using a software system
spread across multiple machines.  In the system a single machine acts as the
master control node (MCN) and orchastrates the process of building a given
package.  Individual machines act as build nodes where a package and its
dependencies are built.  The build farm's MCN manages the building of packages
and their dependencies by assigning the jobs to particular build nodes.  On
each build node the resolve\_deps.pl script manages the process of recursively
building a given package and all its dependencies.  This "build farm" allows
the same package to be built on multiple operating systems and architectures in
a parallel fashion.  This flexible framework allows new platforms and operating
systems to easily be added.  This build farm approach is flexible and requires
little human interaction to build all packages for a particular platform and
operating system. In this way, biopackages is more than just a collection of
software and data packages.  It provides a complete framework for adding and
building new software packages across a variety of platforms and operating
systems with minimal effort.

%%%%%%%%%%%%%%%%%%%%%% DISCUSSION
\section{Discussion}

The Biopackages.net repository is a large collection of software developed that
is relevant to -- and in many cases specifically developed for -- information
management and computational analysis in the biological sciences.  Software is
packaged using the RPM Package Manager and distributed using the Yellow dog
Updater (Modified), both of which are commonly used software management
utilities for Linux-based computers.  At the time of this writing the
repository contains more than 3800 packages  for Fedora Core Linux, Centos %XXX 3800
Linux, and Apple's OS X Darwin 10.4.

A large fraction of the package growth rate can be accounted for by (a) the
recompilation of packages for all platforms, and (b) minor adjustments to
package contents and metadata, analagous to debugging and documentation in the
software development process.  While this type of growth is necessary for the
Biopackages.net repository to remain usable on a variety of platforms, it is
less interesting and less desireable than repository growth to include new
software from the research community, as well as new versions of existing
software.

Substantial effort must be invested to maintain the repository in a useful and
up-to-date state, but it is an overall win for the bioinformatics community if
users buy in to the system, as it allows bioinformatics-specific system
updates to be treated as a service, effectively factored out by delegation to a
group of specialists \cite{nation,soap}.  The amount of effort necessary
to identify and incorporate new software from the community may also be
partially mitigated if users of the Biopackages.net repository reaches a
critical mass.  In this scenario, authors of software are able to attain more
prominence and citation rank by ensuring their software is readily available to
the largest number of users.

\subsection{Licensing}

Academic software is commonly available through a dual-licensing model in which
academic or not-for-profit research use is freely granted, but enterprise-class
users are required to pay for use.  The licensing terms are typically
prominently displayed on the authors' software download pages, and are
difficult to miss.  One potentially adverse effect of using the Biopackages.net
repository through \texttt{yum} is that hundreds of license terms can be
implicitly agreed to with only a few keystrokes.  While this is typically not a
problem for academic users, it may create legal difficulties for users in the
enterprise.  We encourage all users to closely examine all pacakge metadata to
ensure they are entitled to use the software under the terms in which it is
distributed via the Biopackages.net repository.  One possible mechanism to ease
the burden on end users is to partition the repository's software based on
license permissiveness, and is a future goal of the
Biopackages.net project.  Future goals also include contributing to larger and
more general-purpose repositories, such as \url{http://pbone.net/} and
\url{http://atrpms.net}.

\subsection{Non-RPM Systems}

We are also interested in extending the scope of the project beyond operating
systems that use the RPM Package Manager natively.  In our preliminary work in
this area, we focused on Apple MacOS X 10.4 and Cygwin for Microsoft Windows
XP, and used the RPM and \texttt{yum} ports available on these systems.  While it was
possible to use RPM and \texttt{yum} to manage installed software on these systems, it
was neither trivial nor elegant.  The essence of the problem lies in the fact
that at their lower levels, software dependency graphs rely on basic software,
such as the Bourne shell, or the Apache HTTP server.  These basic packages are
available and usable by RPM-installed software on non-RPM-native systems, but
the RPM-based utilities cannot verify dependency integrity.  Our solution to
this problem is to provide virtual packages that claim to provide software
(such as Apache), but actually do nothing and defer to the base system packages
to provide the functionality.

There is also potential to convert Biopackages RPMs into non-RPM formats for
use on a wide breadth of UNIX types. While RPM support has been ported to most
versions of UNIX and Linux, many of these OSes were initially
developed with other binaries in mind. Distributions based on Debian Linux,
for instance, are native to deb packages while Sun's Solaris and many other
UNIXes are native to pkg. Slackware Linux uses .tgz packages, while
Stampede Linux uses .slp packages. Fortunately for maintainers tools has
been developed to convert packages between these many formats. One such
software is Alien which allows conversion between the above named package
types. This creates the prospect of developing Biopackages branches for
different package formats to attract users for whom RPM is not a viable option.

%%%%%%%%%%%%%%%%%%%%%% METHODS
\section{Methods}

\subsection{Package Creation}

The first step in creating a software package is to obtain the package's source
material, or sources, from an upstream provider.  For a software library, this
may be a tarball of C++ files, and for a genomic data set may be several FastA
files.

The second \textit{optional} step is to create patches that modify the package
sources to be compatible with the target platform for installation.
Modifications typically do not change the core functionality of the software,
but customize the software or software installer slightly to be compatible with
the target platform or related software libraries and applications.

The third step is the creation of a metadata file, which for RPM based systems
is called a spec file.  The spec file references the source and patch
files and includes instructions for performing the configuration and
installation of the software.  Additionally, the spec file lists all other
packages that are required in the build- and
install-phase.  The packages depended upon may differ for build and install phases, and may also be conditional by platform .  The spec file also contains other package-related metadata, such as license details and
type of package (library, database, application, etc).

\subsection{Package Version Control and Platform Targeting}

After being initially written, the spec file is then converted to a spec.in
file and imported into the Biopackages.net Concurrent Versions System (CVS)
repository.  The Biopackages.net project abstracts the spec file metadata one
step further for two reasons.

First, the spec file contains a revision metadata field for package version
information, but provides no mechanism for automatic managment of version
numbers.  Likewise, it contains a changelog metadata section for describing
how a package has evolved over time, but does not provide a utility for
tracking changes.  CVS does both automatic revision and changelog tracking, but
in a different format than is required by RPM.  A simple search-and-replace
preprocessing step is sufficient to map the CVS sections for versioning and
changelogs (\texttt{\$Id\$} and \texttt{\$Log\$}) to the corresponding RPM spec
file sections (\texttt{\%revision} and \texttt{\%changelog}).  The
Biopackages.net build system includes a utility to perform this task.

Second, spec file syntax does not allow all portions of the file to be
conditional based on aspects of the platform. With the use of a single spec
file for multiple platforms, it is important to optionally specify different
build and installation requirements for the same package on different Linux
distributions. This is because of the variation of libraries included with a
given package across distributions. As an example, a default Perl on Fedora
Core 2, may include a Perl library that is not included in the default Perl
installation of Fedora Core 5. Therefore it is important to specify this
library as an additional requirement for Fedora Core 5, while unnecessary for
Fedora Core 2. RPM is limited in its ability to include conditional statements
in the build and installation requirments sections of the spec file. Therefore
the solution we have implemented is the inclusion of a platform specific
\texttt{if} control structure in the spec.in file.  This is translated from a
conditional statement to the resulting package dependency when the spec.in file
is converted to a spec file.

\subsection{Package Compilation}

Package compilations are performed on a self-contained Build farm that
contains all of the OS distributions we target. Using VMWare virtualization on a
64-bit AMD based machine, we are able to simultaneously run multiple
32- and 64-bit x86 architecture virtual machine (VM) instances, each with a
different OS and virtual architecture.  This allows us to build packages for
many different platforms simultaneously.

The RPM construction process uses the Sun Grid Engine (SGE) cluster management utility to
orchestrate construction of a given \textit{target} package for multiple platforms, and is illustrated in ~\ref{fig:bp_buildfarm}.
This process is initiated by iterating over all platforms targeted for construction, and interacting with the
SGE Master Control Node (MCN) to build those packages.  MCN manages all VMs and ensures
that each request to build \textit{target} for a particular platform is delegated to the correct VM.  Once
delegated to a VM, package construction process can be divided into three phases:
staging, building, and cleanup.  These are illustrated in Figure \ref{fig:In2rpm}.

In the staging phase, \textit{target}'s metadata is converted from the CVS \texttt{RCS} to \texttt{rpmbuild}
format, and all packages that are required to build \textit{target} are identified and installed.  This
process is coordinated by the \texttt{resolve\_deps.pl} script.  It begins by parsing \textit{target}'s
metadata to identify all packages that are depended upon for build or install of \textit{target}.  Each of
these depencies in turn has its dependencies identified, and so on, recursively.  When the terminal
packages of the dependency graph are reached, the recursive process traverses back toward
\textit{target}, and at each step installs the dependency to ensure that all requirements in the graph
have been installed.

In the building phase, \textit{target}'s metadata is used to construct the RPM package.  This is
done with the \texttt{rpmbuild} utility, which typically invokes several other scripting tools and
software compilers to prepare the \textit{target} software for the VM's platform.

Finally, cleanup phase takes place.  All packages that were installed to satisfy the dependency are
uninstalled.  This restores the build system to a ``clean'' state ready for the next scheduled build.
The rpm file for \textit{target} is archived to a repository where all complete packages are moved to shared storage and
made available to the public via a web server.  Log files created during the staging and build phases
are also archived to shared storage for post-processing.  Examples of post processing include the generation of build
reports, identifying VM-specific problems, and detection and re-queueing of failed builds.

After all packages have been iterated over and all requests scheduled with the MCN have completed,
the packages in the RPM repository are indexed using the \texttt{createrepo} command.  This enables
\texttt{yum} clients to connect to the web server and install packages.

\subsection{Package Deployment}

Built RPMs are automatically written to a shared NFS volume that is
accessed by the Biopackages web server. Packages are initially placed into a
``testing'' repository corresponding to distribution and architecture. After
enough testing of package stability, RPMs are migrated to the ``stable''
repository. On the web server, a nightly cron job automates the creation of
\texttt{yum} and \texttt{apt} headers for all platforms of both the stable and testing
biopackages repositories. From this server, packages are served over HTTP and
NFS to both local and external clients that make requests.

\subsection{Package Installation}

Once headers are generated on the web server, the user has the choice of
several installation methods for a given RPM. The native Red Hat method is to
use RPM to perform the installation directly from the web server, or from a
local disk the package has been downloaded to. While this allows automated
downloading and installation of a package, RPM is limited in its inability to
perform dependency resolution and therefore requires manual installation of
dependencies. Therefore an easier installation method is through the use of \texttt{yum}
or \texttt{apt}. \texttt{yum} allows for easy installation from a software repository by locating
and downloading the desired packages and all of its dependencies in one
automated transaction. \texttt{apt} serves a similar purpose as \texttt{yum}, originally being
ported from Debian Linux for use with RPMs. Therefore some users are more
comfortably with and prefer the syntax and capabilities of \texttt{apt}.

Another decision to be made by the user is if they would like to use ``testing''
RPMs. \texttt{yum} and \texttt{apt} make it possible for a user to specify which repositories
they prefer, and which repositories are only to be used with manual input. This
allows a user to disable regular use of the ``testing'' repository, while allowing
manual overriding when a package is present only in the ``testing'' repository.

\subsection{Package Quality Assurance}

In maintaining a software repository, a crucial issue is adopting a method of
ensuring stability and quality of packages and identifying problematic
software. A two tiered repository approach helps separate production tested
packages from more newly built RPMs. Newly built RPMs are placed into the
Testing repository, where they are served to both local and external users for
functional testing to be performed. This allows assurance of proper
documentation, configuration files, log files, runtime scripts and other
software functionality before a package is deemed ``production ready'' and
migrated to the ``stable'' repository. It is in this testing period that bugs are ideally
discovered and reported by users and developers, to allow fixes to be made
before the package is moved to the ``stable'' repository. Most Testing packages are
considered to be mostly stable and are immediately implemented on local compute
nodes, however a separate Stable branch adds an extra level of quality control
for production environments.

\section{Figure Captions}

\subsection{Figure 7.1}

Packages are represented by
boxes, with grey provided by the OS distribution, and white provided by
Biopackages.net.  Packages may rely on other packages, either at build- or
install-time.  Solid head dashed lines indicate a n installation dependency,
hollow head dashed lines indicate a build dependency, solid head solid lines
indicate both an installation and build dependency.

\subsection{Figure 7.2}

A spec.in file is preprocessed to form
a spec file, which is processed by the \texttt{rpmbuild} utility along with the
package source files to form architecture-specific packages and a source
package that contains the .spec file and source files.

\subsection{Figure 7.3}

A controlling script
observes the state of the package repository, and spawns jobs to build missing
packages via a master/slave cluster managmement system.  Packages are build and
uploaded to a shared filesystem, where they are then made available over the
network to clients wishing to install software.


% %%%%%%%%%%%%%%%%%%%%%% ACKNOWLEDGEMENTS
% \section{Acknowledgements}
% 
% We acknoledge the following individuals for their help in the development,
% documentation, testing and maintenance of the software and systems described
% here: Patrick Alger, Andrew Helsley, and Victor Ruotti.  Additionally, we thank
% Scott Cain, Steve Chervitz, and Todd Harris for discussion related to the
% design of the project architecture.  We thank Ling Lee for designing the
% Biopackages.net website.  This work was supported by NIH grants XXXXXXXXXXXXX.

%%%%%%%%%%%%%%%%%%%%%% FIGURES
\pagebreak
%\section{Figures}

\begin{figure}[p]
\includegraphics[width=12cm]{figures/Bp_depgraph.png}
\caption{Partial package dependency graph.}
\label{fig:bp_depgraph}
\end{figure}

\begin{figure}[p]
\includegraphics[width=12cm]{figures/In2rpm.png}
\caption{Package compilation process.}
\label{fig:In2rpm}
\end{figure}

\begin{figure}[p]
\includegraphics[width=12cm]{figures/Bp_buildfarm.png}
\caption{Biopackages.net build farm architecture.}
\label{fig:bp_buildfarm}
\end{figure}


%%%%%%%%%%%%%%%%%%%%%% REFERENCES
\pagebreak
\bibliographystyle{plain}
\bibliography{biopackages_3-16-07}


