\section{DNA Microarrays}

The first human genome sequence, completed in 2001, ushered in a new era of
biology.  This effort accelerated the development of biochemical assay and
information technologies.  As a result, biologists are now able to ask
questions that were previously considered intractable.  One example of a
breakthrough in assay technology resulting from genome sequencing efforts is
the DNA microarray, a high-throughput measurement device which enables
individual scientists to rapidly and simultaneously interrogate the RNA
concentration levels of virtually all genes in the human genome for a single
biological source.  The microarray is one of the most commonly used of
post-genomic high-throughput assay technologies, as is indicated by the number
of abstracts in PubMed containing the word ``microarray''.  It has been widely
adopted in the biological sciences and been used to address a large number of
research questions across a diverse set of subdomains in biological research.

Experiments that collect data using microarray technology can be divided into
several distinct categories, based on their overall goal and tactics employed
in analysis of the data.  These include: classification problems (supervised
learning) \cite{Dudoit2003ICM}, clustering problems (unsupervised learning)
\cite{Azuaje2003cge,Stanford2003cac}, gene co-regulation and function
identification studies \cite{PMID_12413821}, differential gene expression
studies \cite{PMID_15843092,PMID_15867208}, time-course and dose-response
studies \cite{PMID_12443997}, gene pathway and regulatory network studies
\cite{PMID_16216773,PMID_16825123}, drug discovery and toxicology studies
\cite{PMID_16512775,PMID_16700885,PMID_16880944,PMID_17195470}, clinical
diagnostics \cite{PMID_16918486}, and sequence-variation studies
\cite{PMID_17265721}.

A successful microarray experiment is one which is able to provide insight into
the hypothesis for which it was designed.  Insights gained by experimentation
are hard-earned, and require careful planning and execution of multiple key
steps.  The initial steps of feature selection, microarray design, microarray
fabrication, experiment design, and biological sample processing, and image
acquisition are described generally in Sections
\ref{ArrayDesign}-\ref{Hybridization}.  A more exhaustive discussion of
protocol surrounding these steps is given in \cite{wit2004}.  Following their
acquisition, images are quantified, stored, distributed, analyzed, and
analytical results shared.  Challenges and findings related to these latter
procedures are the subject of \emph{\dbthesis}.  They are thoroughly discussed
in Sections \ref{Data Processing}-\ref{Informatics}.  The entire process, from
microarray design to data analysis has been formally described as an object
model known as the Microarray Gene Expression Object Model (MAGE-OM)
\cite{mage}.  In this document I have borrowed vocabulary defined in the
MAGE-OM for clarity and consistency.  The MAGE-OM itself is discussed in
Section \ref{MAGE}.

\section{DNA Microarray Fabrication and Assay Protocols}
\label{Assay}
There are two basic types of gene microarray technologies currently at the
disposal of researchers: single-channel and multi-channel microarrays.  The
methods used and form of measurements made for these two types differ, but the
research questions that are addressable using either type are the same.  In
both systems, DNA is immobilized on a fixed surface in a specific and
reproducible pattern.  This is the \emph{microarray}, sometimes abbreviated as
an \emph{array}.  The pattern that allows distinct regions on the
two-dimensional space of the surface to be mapped to the specific DNA sequences
immobilized at that address is known as a \emph{array design}.  Each distinct
region on the array is known as a \emph{feature} and the DNA sequence present
at a feature is called a \emph{probe}.  The source of the probe that is
immobilized at the feature can be a cDNA clone deposited onto the microarray,
or oligonucleotides that are synthesized on the microarray \emph{in-situ} using
photochemistry and light masks \cite{fodor,lipshutz}.

\subsection{Array Design and Fabrication}\label{ArrayDesign}

The first steps in performing a microarray experiment is deciding which genes
to assay on the microarray, how many genes to add, and how to arrange the genes
on the microarray.  These steps are collectively known as \emph{microarray
design}.

In earlier times it was commonplace to manufacture multi-channel microarrays in
the laboratory using a form of printer that deposited oligonucleotides onto a
glass slide.  This is no longer the case, as microarray production has been
industrialized and is now done on a scale and level of precision that requires
specialized apparata that are not practical to host in a laboratory
environment.  However, the design principles remain the same.  The array
designer needs to select the sequences that are to be assayed by the
microarray.  These can correspond toregions that are transcribed to RNA,
non-transcribed regions, or even oligonucleotides not expected to be observed
in the biological source material to be hybrized.  Historically, limitations in
manufacturing technology were very restrictive, and limited the number of genes
that could be assayed simultaneously.  Current technology is sufficiently
advanced to allow all genes to be simultaneously assayed for most organisms, so
the question of which and how many genes to add is no longer relevant.
However, as anyone who has performed and experiment will aver, the results
obtained from the same experimental procedure performed twice will yield
different results.  This also holds true for measurements made with
microarrays.  There are many potential sources of variance in the gene
measurements, and a good array design measures as much of this measurement
variance as possible.  Estimation of variance on a microarray essentially means
making multiple measurements of the same gene.  Each gene can be measured at
multiple distinct locations along its sequence.  Further, each location can be
measured multiple times.  A good array design uniformly distributes replicates
across the array surface to compensate for local perturbations on the array
surface itself.

Once the sequences to place onto the microarray have been determined, the
sequences are either placed onto the physcial substrate using printing
technology, or synthesized directly onto the substrate using photolithography
\cite{fodor,lipshutz}.  Fabrication and microarray design are interdependent,
in that improvements in manufacturing techniques allow for the fabrication of
more powerful devices.  Thus, this area of microarray technology is evolving
especially rapidly and any specific details of the synthesis I can describe
here will quickly be outdated.

Metadata are maintained that map the physical position of features in the
arraydesign of the microarray back to the sequences they were designed to match
during hybridization (Section \ref{Hybridization}).  These are used by the data
analyst after all data are collected to detect changes in gene expression that
correspond to changes in experimental factors.

\subsection{Hybridization}\label{Hybridization}

Once a microarray has been fabricated according to the array design, a solution
of DNA fragments, or \emph{targets}, can be labeled with a fluorescent dye and
\emph{hybridized}, or allowed to chemically anneal to the microarray's probes.
The property of reverse complementarity between subsets of the solvent and
immobilized DNA drives the solvent DNA targets to localize near to their
corresponding immobilized probes.

After hybridization, the amount of target DNA hybridized to each feature on the
array is assayed.  A laser is used to excite the fluorescent dye associated
with the targets, and a digital camera is used to measure the amount of
fluorescence present at each feature, and at the specific wavelength of
fluorescence known as a \emph{channel}.  The amount of fluorescent light is
assumed to be a monotonic and increasing function of the amount of target DNA
hybridized to that feature.  In the case of single-channel assays, only a
single fluorescent dye is used.  In a multi-channel assay, multiple target
solutions are hybridized to the array, each labeled with a different
fluorescent dye.  Each digital image of a microarray's fluorescence pattern for
a single channel is known as an \emph{image}, and the process of observing and
recording this image is called \emph{acquisition}.

For a single-channel assay, the acquired image is processed to produce a matrix
of numbers that has the same dimensions as the array design, and whose values
correspond to the absolute intensity of light present at each feature.  A
similar matrix is produced for a multi-channel assay as well, but because of
the competitive hybridization between the multiple labeled targets, the values
reported are not absolute, but rather the intensity of each channel relative to
all other target channels at a given feature.  Typically one of the channels
corresponds to a \emph{reference sample} that is a common denominator to
several arrays that will be analyzed as a set.  This allows the relative values
for the non-reference samples on each array to be compared.

For both single-channel and multi-channel arrays, the next step is
\emph{quantification}, a series of processes designed to convert the ``raw''
fluorescence intensities acquired from the scanner into values that are usable
in higher-level data analyses such as classification.  Methods for
quantification of microarray data are an area of intense interest
\cite{mas5,affy4,mbei,rma,vsn,gcrma,affyplm,seo,affybench} because they can
give very different estimates of the quantity of DNA present for a given
feature/channel, and thereby have a large effect on the biological conclusions
that can be drawn from analyses of those estimates.  Quantification methods are
discussed in greater detail in Section \ref{Data Processing}.


\section{DNA Microarray Experiment Design}
\label{Experiment Design}
Scientific inquiries that use microarray technology to address research
questions are referred to as \emph{microarray experiments}.  The majority of
these experiments can be divided into two classes: pattern detection and
predictive modeling.  These are described in greater detail in Sections
\ref{Pattern Detection}-\ref{Predictive Modeling}, and specific examples of
these experimental classes can be seen in \cite{Dudoit2003ICM, Azuaje2003cge,
Stanford2003cac, PMID_12413821, PMID_15843092, PMID_15867208}.  The method
descriptions given here are desribed in terms of the characterization of
biological samples, but it is important to bear in mind that the
samples-by-genes data matrices can be transposed and that these same methods
can be used for the characterization of genes across multiple phenotypic
conditions.  Indeed, this is also an area warranting further exploration given
that gene-centric analyses are not stymied by the ``curse of dimensionality'',
discussed in greater detail in Section \ref{Feature Selection}.

\subsection{Pattern Detection}\label{Pattern Detection}

The first class of experiment is data-driven, in which an \emph{a-priori}
hypothesis is not explictly stated and sample metadata in the form of labels
are not provided.  This data driven method is sometimes also called
\emph{pattern detection} because experiments employing it focus mainly on
identifying biologically interesting and previously unobserved correlations
between factors of the experiment \cite{Dubitzky2003IMD}.  A common type of
pattern detection experiment seen in the literature examines biological samples
that are indistinguishable using lower-resolution assay technologies to see if
the samples can be broken into sub-groups by collecting more data using the
higher-resolution microarray.   A study by Freije, \emph{et al.} used this
approach to identify novel subclasses of malignant gliomas
\cite{PMID_15374961}.

\subsection{Predictive Modeling}\label{Predictive Modeling}

In the second general class of experiment seen in the literature, the emphasis
is on constructing a function that is able to \emph{classify}, or predict class
labels for unlabeled samples.  This method is sometimes called \emph{supervised
learning} or \emph{predictive modeling} \cite{Dubitzky2003IMD}.  A predictive
modeling study begins with a group of samples for which assay data as well as
class labels are available.  This initial set of labeled samples is divided
into two sub-groups, \emph{training data} and \emph{test data}.  The
training data are processed using a classification algorithm to build a model
of the relationship between class label and assay data.  One major area of
concern in the construction of the model and the application of that model to
unlabeled samples is the number of features considered.  Feature selection is
discussed in greater detail in Section \ref{Feature Selection}.

The performance of the classification model produced using the training data is
typically assessed and iteratively optimized to minimize error using $K$-way
cross validation for all class labels $K$.  However, in some cases class labels
$K$ are nested meaning that the class $k_{i}$ can be a refinement of the
class $k_{i-1}$.  In these hierarchical classification models, optimization and
error minimization consider the graph structure \cite{pachinko}.  Once a
satisfactorily low error rate has been achieved, the predictive power of the
classification model can be estimated by using it to predict labels for the
test data that were set aside prior to model building.
% A study by Day and Dong in Chapter \classchapter demonstrates the usage of
% both flat and hierarchical classification methods.

\subsection{Feature Selection}\label{Feature Selection}

A prominent part of model building for the analysis of microarray data is
\emph{feature selection}, or the process of identifying which features of
observations are relevant in the assignment of class labels.  Selected features
correspond to the microarray feature measurements, or some summarization
thereof (Section \ref{Summarization}).

Feature selection has been essential because the analytical methods applied to
microarray experiments bear the ``curse of dimensionality''
\cite{Bellman_1957}.  In a typical experiment the number of samples is small
($1{\times}10^1 \dots 1{\times}10^3$), while the number of measurements made on
each sample is relatively large ($1{\times}10^3 \dots 1{\times}10^6$).  Because
of this it is possible to correctly partition labeled data on one or multiple
subsets of all observed features.  Using a subset that robustly captures the
sample labels is desireable because it increases classification accuracy and
reduces the computation required required to calculate a new sample label, as
many classification algorithms scale exponentially with the number of features
considered \cite{john94irrelevant}.

However, the work described in Chapter \celsiuschapter presents a mechanism for
aggregation and analysis of much larger numbers of samples than are commonplace
in experiments to date.

\subsection{Label Encoding}\label{Label Encoding}

In both the pattern detection (Section \ref{Pattern Detection}) and predictive
modeling (Section \ref{Predictive Modeling}) classes of microarray studies, a
class label from one or more multiple orthogonal experimental factors is
attached to a sample.  Class labels are typically ``flat'', meaning that there
is no hierarchical structure and that the distance between any two classes is
uniform.  However, It is well-established that such a flat representation is
not suitable for representing gene annotation \cite{go}, and it has been
demonstrated that statistical analyses that consider the structure of
relationships between annotations bear more power than their flat counterparts
\cite{ease,gocluster,pachinko}.  Thus, analyses that explicitly model the
relationship types and distances between class labels of samples are expected
to become more commonplace as open, community-supported, standard structures
for encoding sample annotations mature \cite{so, obo, mpath, cl, mo, ma, mpath,
mp}.  Encoding metadata using open standards also has implications with regard
to integration of results from multiple studies, or otherwise exchanging data.
These aspects are discussed in Section \ref{Protocol}.


\section{DNA Microarray Data Processing}
\label{Data Processing}
%XXX Introduce the section on data processing...

%%% XXX old outlining.  can probably be tossed
%\subsection{Quantification}
%
%Microarray quantification, also known as \emph{pre-processing}, is the
%process of estimating the quantity of each gene in the sample that was assayed
%in the hybridization step of the microarray experiment.
%
%\subsubsection{Image Acquisition}
%\subsubsection{Background Correction}
%\subsubsection{Normalization}
%\subsubsection{Perfect Match Correction}
%\subsubsection{Summarization}
%\subsection{Scalability Concerns In Microarray Data Processing}
%%% XXX end old outlining...

citations to merge in:
liwong                  \cite{mbei,dchip}
RMA                     \cite{rma}
genelogic               \cite{genelogic}


\subsection{Quantification}
\label{Quantification}

Microarray pre-processing, also known as \emph{quantification}, is the
process of estimating the quantity of each gene in the sample that was assayed
in the hybridization step (Section \ref{Hybridization}) of the experiment.
Quantification can be broken down into five distinct sub-procedures, executed
in the following order: image processing, background correction, normalization,
PM correction, and summarization.

\subsubsection{Image Processing}
\label{Image Processing}

\subsubsection{Background Correction}
\label{Background Correction}

Background correction is a statistical procedure that estimates and removes low
levels of noise on the microarray.  Background noise can have many sources.

The simplest and most common source of background noise is optical.  It can be
caused by general cross-hybridization of target to all probes, mis-calibration
of the microarray scanner's photo-sensor, and diffused or reflected light from
the laser used to excite the fluorescent dyes.  Optical noise can be estimated
by measuring the level of fluorescence from featureless regions of the
microarray and negative control probes that are not reverse-complementary to
any sequences in the hybridization mixture.  These measure background-level
reflected light and the level of non-specific hybridization, respectively.

Manufacturing and hybridization artifacts, such as surface scratches and salt
residues, are another source of noise.  A simple form of location-based
background correction is descibed in the Statistical Algorithms Description
Document \cite{affy:tech:2002}.   Briefly, the chip is broken into a $4x4$
grid of 16 rectangular regions.  The lowest 2\% of each region's probe
intensities are used to compute a background value for that region.  Each probe
(PM and MM) is then adjusted based upon a weighted average of the backgrounds
for all regions. The weights are based on the distances between the location of
the probe and the centroids of all regions.  More sophisticated methods attempt
to detect areas of the microarray containing high levels of manufacturing and
hybridization noise.  Noisy areas can be identified because the probes located
there will be outliers relative to probes for the same target located elsewhere
on the microarray.  Probes in these areas are considered unreliable and are
either given a very low weight parameter or are removed from normalization
(Section \ref{Normalization}) and other downstream processing (Sections \ref{PM
correction}-\ref{Summarization}) altogether \cite{affyplm}.

Newer, multi-array background correction methods have leveraged existing data
to build a models of how background noise is generally distributed.  The gcRMA
model \cite{gcrma} includes a parameter the sequence composition of each
probe, while other models such as those used in the RMA and MBEI
\cite{rma,bioc} methods only include a parameter for concordant each probe is
with other probes in the same set.  The RMA background correction method is the
\i{de facto} standard, and corrects perfect match (PM) probe intensities by
using a global model for the distribution of probe intensities. The model is
suggested by looking at plots of the empirical distribution of probe
intensities.  In particular the observed PM probes are modeled as the sum of a
normal noise component N (Normal with mean $\mu$ and variance $\sigma^2$) and a
exponential signal component S (exponential with mean $\alpha$). To avoid any
possibility of negatives, the normal is truncated at zero. Given we have O the
observed intensity, this then leads to an adjustment, given in Equation
\ref{rmabg}:

\begin{equation}
\label{rmabg}
E\left(s \lvert O=o\right) = a + b \frac{\phi\left(\frac{a}{b}\right) - \phi\left(\frac{o-a}{b}\right)}{\Phi\left(\frac{a}{b}\right) + \Phi\left(\frac{o-a}{b}\right) - 1 }
\end{equation}

where $a =  s- \mu - \sigma^2\alpha$ and $b = \sigma$. Note that $\phi$ and
$\Phi$ are the standard normal distribution density and distribution functions
respectively.  Note also that MM probe intensities are not corrected by either
of these routines \cite{rma,bioc}.

Multi-array background correction methods are able to detect background noise
due to the manufacturing and hybridization artifacts described above, but the
size of the aray artifact can be as small as a single feature.  This should in
principle do a better job of noise estimation.  A major drawback to multi-array
background noise models is that the noise estimates are only valid in the
context of the co-processed set of microarrays.  This is because the noise
estimates are derived from parameter estimates specific to that set of
microarrays.  While this is not a problem for small-scale analysis on
individual experiments, it creates difficulties when merging data from multiple
experiments because all microarrays will need to be re-processed to re-fit the
parameters of the noise model.  This re-processing problem can become
intractable for background correcting a very large number of arrays, and is
discussed in greater detail in Section \ref{Scalability}.

%%%%%%%%DEAD TEXT
%  Typical sources of probe-independent noise
%are the miscalibration of the photo-sensor used in Section \ref{Image
%Processing}, diffuse laser reflection from the microarray, debris and salt left
%over from hybridization (Section \ref{Hybridization}), and interaction between
%probes and non-target sequences.  Noise falls into two general categories:
%probe-independent and probe-specific.
%
%The level of probe-independent noise is estimated by measuring the level of
%fluorescence from featureless regions of the microarray and negative control
%probes that are not reverse-complementary to any sequences in the hybridization
%mixture.  These measure background-level reflected light and the level of
%non-specific hybridization, respectively.  Probe-independent background
%correction is generally done during image processing because the
%probe-independent noise tends to be spatially localized or uniformly
%distributed, although emerging methods include probe sequence in modeling
%background noise and correct each probe independently
%\cite{mbei,PMID_12582260}.  The level of probe-dependent noise is
%estimated by measuring the intensity of probes that differ only by a single
%nucleotide from their target sequence.  The idea to using these so-called
%\emph{mismatch probes} (MM) is that because their sequence is nearly identical
%to the sequence of the \emph{perfect match} (PM) probe that any difference
%between hybridization between a set of PM/MM pairs is almost entirely
%sequence-specific and thus provides a probe set-specific measure of noise
%\cite{rma,bioc}.

\subsubsection{Normalization}
\label{Normalization}

After correcting for background noise (Section \ref{Background Correction}),
microarrays are normalized.  The purpose of normalization is to transform the
distribution of microarray measurements so that properties of the
distribution of measurements match expectations.

The most simple form of microarray normalization is a linear scaling.  The
Affymetrix MAS 5 algorithm \cite{mas5} performs linear scaling by (1) setting
aside the top and bottom 1\% of measurements as outliers, adjusts the mean of
the remaining measurements to a constant value, then multiplies each
measurement, including the outliers, by the factor used to adjust the mean.

some properties of the normalized measurements fit an expected distribution.

\subsubsection{PM correction}
\label{PM correction}

\subsubsection{Summarization}
\label{Summarization}

The last step in microarray data preprocessing is to combine the measurements
from all probes in a probe set into a single value.  This procedure is called
\emph{summarization}.  The simplest summarization algorithm, called ``average
difference'' \cite{affy4} computes the mean of difference between each PM/MM
probe pair (Equation \ref{avgdiffsummary}),

\begin{equation}
\label{avgdiffsummary}
y_{k} = I_i^{-1}{\sum_{i=1}^{I_k} |PM_i-MM_i|}
\end{equation}

where the probe set $k$ has $PM$ perfect match and $MM$ mismatch probe pairs $i
= 1,\dots,I_k$.

Summarization methods parallel background correction and normalization methods
in that there are two varieties, the single-array methods and the multi-array
methods.  ``Average difference'' is an example of the former.  Multi-array
methods consider the distribution of probe measurements across all arrays, and
in some cases assign an array-specific parameter used to compute the probe set
summary.  The summarization component of the MBEI method introduced by Li and
Wong \cite{mbei,dchip} is given in Equation \ref{mbeisummary},

\begin{equation}
\label{mbeisummary}
y_{ij} = \phi_i \theta_j + \epsilon_{ij}
\end{equation}

where $y_{ij}$ is $PM_{ij}$ or the difference between $PM_{ij}-MM_{ij}$. The
$\phi_i$ parameter is a probe response parameter and $\theta_j$ is the
expression on microarray $j$.

The summarization component of RMA pre-processing \cite{rma} performss a
multi-array linear fit to data from each probe set.  Specifically, for probe
set $k$ with $i=1,\dots,I_k$ probe pairs and microarrays $j=1,\dots,J$ the
model given in Equation \ref{medianpolish} is fit,

\begin{equation}
\label{medianpolish}
\log_2\left(PM^{(k)}_{ij}\right) = \alpha_i^{(k)} + \beta_j^{(k)} + \epsilon_{ij}^{(k)}
\end{equation}

where $\alpha_i$ is a probe effect and $\beta_j$ is the $\log_2$ expression
value, and the method is known as \emph{median polish}, named after Tukey's
algorithm used to perform the calculation.

It is noteworthy that summarized probe set values from all popular multi-array
summarization methods, including those described here, are dependent upon the
probe set and microarray effect parameters calculated as part of the model fit.
While this is not a problem theoritically, it introduces unique challenges in
the implementation of a pre-processing pipeline for a large number of arrays.
This is discussed in greater detail in Section \ref{Scalability}.

%\subsubsubsection{Normalization}
%%%%%%%%%%%%%%%%%%%%%%%%%%%%%%%%%%%%%%%%%%%%%%%%%%%%%%%%%<QUOTE>

You can see the background correction methods that are built into the package
by examining the variable \verb+bgcorrect.method+.

<<>>=
normalize.AffyBatch.methods
@
The Quantile, Contrast and Loess normalizations have been discussed and compared in \cite{affybench}.

\subsection{quantiles/quantiles.robust}

The quantile method was introduced by \cite{affybench}. The goal is to give each chip the same empirical distribution. To do this we use the following algorithm where $X$ is a matrix of probe intensities (probes by microarrays):

\begin{enumerate}
\item Given $n$ microarray of length $p$, form $X$  of dimension $p \times n$  where
each microarray is a column
\item Sort each column of $X$ to give $X_{\mbox{sort}}$
\item Take the means across rows of $X_{\mbox{sort}}$ and assign this mean to each element in the row to get $X'_{\mbox{sort}}$
\item Get $X_{\mbox{normalized}}$ by rearranging each column of $X'_{\mbox{sort}}$ to have the same ordering as original $X$
\end{enumerate}

The quantile normalization method is a specific case of the transformation $x'_{i} = F^{-1}\left(G\left(x_{i}\right)\right)$, where we estimate $G$ by the empirical distribution of each microarray and $F$ using the empirical distribution of the averaged sample quantiles.  Quantile normalization is pretty fast.

The \emph{quantiles} function performs the algorithm as above. The \emph{quantile.robust} function allows you to exclude or down-weight microarrays in the computation of $\hat G$ above. In most cases we have found that the \emph{quantiles} method is sufficient for use and \emph{quantiles.robust} not required.

\subsection{loess}

There is a discussion of this method in \cite{affybench}. It generalizes the $M$ vs $A$ methodology proposed in \cite{Dudoit:2002} to multiple microarrays. It works in a pairwise manner and is thus slow when used with a large number of microarrays.

\subsection{contrasts}

This method was proposed by \cite{astr:2003}. It is also a variation on the  $M$ vs $A$ methodology, but the normalization is done by transforming the data to a set of contrasts, then normalizing.

\subsection{constant}

A scaling normalization. This means that all the microarrays are scaled so that they have the same mean value. This would be typical of the approach taken by Affymetrix. However, the Affymetrix normalization is usually done after summarization (you can investigate \verb+affy.scalevalue.exprSet+ if you are interested) and this normalization is carried out before summarization.

\subsection{invariantset}

A normalization similar to that used in the dChip software \cite{mbei,dchip}. Using a baseline microarray, microarrays are normalized by selecting invariant sets of genes (or probes) then using them to fit a non-linear relationship between the ``treatment'' and ``baseline'' microarrays. The non-linear relationship is used to carry out the normalization.

\subsection{qspline}
This method is documented in \cite{workman:etal:2002}. Using a target microarray (either one of the microarrays or a synthetic target), microarrays are normalized by fitting splines to the quantiles, then using the splines to perform the normalization.
%%%%%%%%%%%%%%%%%%%%%%%%%%%%%%%%%%%%%%%%%%%%%%%%%%%%%%%%%</QUOTE>

%\subsubsubsection{PM Correct Methods}
%%%%%%%%%%%%%%%%%%%%%%%%%%%%%%%%%%%%%%%%%%%%%%%%%%%%%%%%%<QUOTE>
<<>>=
pmcorrect.methods
@
\subsection{mas}

An \emph{ideal mismatch} is subtracted from PM. The ideal mismatch is documented by \cite{affy:tech:2002}. It has been designed so that you subtract MM when possible (i.e. MM is less than PM) or something else when it is not possible. The Ideal Mismatch will always be less than the corresponding PM and thus we can safely subtract it without risk of negative values.

\subsection{pmonly}

Make no adjustment to the pm values.

\subsection{subtractmm}

Subtract MM from PM. This would be the approach taken in MAS 4 \cite{affy4}. It could also be used in conjunction with the Li-Wong model.
%%%%%%%%%%%%%%%%%%%%%%%%%%%%%%%%%%%%%%%%%%%%%%%%%%%%%%%%%</QUOTE>

vsn			\cite{vsn}
genelogic		\cite{genelogic}

A few sentences to dismiss 2-channel microarrays.  There is a parallel set of conceptually identical problems, but the details for how to solve them differ.
Need data consistency.  How to get it


\section{DNA Microarray Information Systems}
\label{Informatics}
\subsubsection{Data Modeling}
\label{Data Modeling}

In order to be able to perform analyses on the results of a DNA microarray
experiment, the assay data collected must be structured and stored.  This is
also true of the experimental \emph{metadata}, or description of the
experimental design and procedures.  Principles of scalability and
mass-production prescribe that the encoding of these data be uniform, meaning
that the structure used to encode the experiment must be sufficiently flexible
and descriptive to capture the full range of experiments that can be conceived.

The \emph{MicroArray Gene Expression} (\emph{MAGE}) Group was established by
the \emph{MicroArray Gene Expression Data Society} (\emph{MGEDS}, sometimes
also referred to as \emph{MGED}) to develop a standard for the representation
of microarray data, with the intent of making those data exchangeable between
different information systems.  The work of the MAGE Group can be divided into
two types of projets: those dealing with the syntax used to concretely
represent microarray data, and those dealing with the symantics used to
describe the microarray experimental materials and processes, produced data,
and data derived from processing.

The semantic developments of the MAGE Group are more important for \dbthesis
than the syntactic developments.  This is because semantics are abstract and
generally useful for encoding experimental information, while the syntactic
developments of the MAGE Group have optimized for the purposes of data
interchange using technologies such as the \emph{eXtensible Markup Language}
(\emph{XML}) that are not well-suited for systems whose primary purpose is to
store and retrieve large volumes of quantitative data.

Application of the MGED semantic technologies are discussed in other sections
of this document.  There are two major semantic developments from MGEDS.  The
first of these is the \emph{MAGE Object Model}, or \emph{MAGE-OM}.  The purpose
of the MAGE-OM is to provide a standard set of classes that can be used to
represent any object or process that may be included in, or referenced from, a
microarray experiment.  The model employs concepts common to object-oriented
software design and knowledge engineering, such as subclass/superclass
relationships between object classes, the notion of abstract classes, and the
possibility for directional and cardinal relationships between objects.  The
specification of the MAGE-OM was developed using the \emph{Unified Modeling
Language} (\emph{UML}), a standard technology used by knowledge and software
engineers for the composition of object models.  Fitting microarray experiment
data into the MAGE Object Model is described in greater detail in Section
\ref{Experimental Metadata}.  Encoding specific information about objects under
investigation or that are used to facilitate the conduct of an experiment are
discussed in Section \ref{Object Metadata}.

Unique syntax-related challenges that have arisen as part of \dbthesis are
also presented.  Internal data representation and scalability issues are
discussed in Section \ref{Storage} and interoperability concerns are discussed in
greater detail in Section \ref{Protocol}.

\subsubsection{Experimental Metadata}
\label{Experimental Metadata}

As a concrete example of how an experiment might be encoded into a MAGE-ML
document, consider the now-classical study of leukemias by Golub, et al
\cite{golub}.  Briefly, this study is has both pattern-detection and
predictive-modeling methodology (Sections \ref{Pattern
Detection}-\ref{Predictive Modeling}), and describes a method for identifying
features that discriminate between two classes of leukemia and how they can be
used to identify the class of previously unlabeled cancer samples.

Encoded into the MAGE-OM, the initial cancer samples in Golub, et al
\cite{golub} are represented as \emph{BioSource} objects, a class used for
biological material prior to any treatment.  Each BioSource object then goas
through a series of modifications, ultimately resulting in a
\emph{LabeledExtract} object that represents the fluorescently-labeled cRNA
that is hybridized onto a microarray.  In the series of modifications, a
combination of a \emph{BioSample} object and one or more \emph{Treatment}
objects is used to represent the transformed Biosource object.  Further, each
LabeledExtract refers back to the object from which it was derived all the way
back to the BioSource object so that the full path of derivation via treatment
is modeled.

To represent the microarray hybridization, a \emph{BioAssay} object is created.
The BioAssay is a central connection point in the object model.  It refers to
an \emph{Array} object representing the microarray itself, the LabeledExtract
that is hybridized, and to one or more \emph{Factor} objects.  The factor
objects are in turn related to a network of other objects that encode the
design and variables used in the microarray experiment.  The Array object is
associated to a series of other objects that describe the microarray itself,
including the information about specific sequences and their physical locations
on the microarray, as well as information about the grouping of features on the
array used as reporters for a common cRNA target sequence.  Further, each
LabeledExtract is associated with a \emph{Channel} object.  The Channel is used
to link a specific LabeledExtract to a specific \emph{Image} object that
results from the scanning of the hybridized array.

The Image object is combined with a \emph{FeatureExtraction} object to produce
a \emph{BioAssayData} object.  This association of objects represents that
transformation of the microarray image acquired by the scanner (Section
\ref{Assay}) into a numerical form that can be processed (Section \ref{Data
Processing}) for further analysis.  BioAssayData objects may also be derived
from other BioAssayData objects, similar to the way BioSource, BioSample, and
LabeledExtract objects may be related.  This is how the MAGE-OM represents
arbitrary data transformations, and is sufficient for describing microarray
data pre-processing (Section \ref{Data Processing}), as well as additional
downstream summarizations or other transformations of these data.

\subsubsection{Object Metadata}
\label{Object Metadata}

Objects in the MAGE-OM may have attributes attached to them to provide
more specific detail about the microarray experiment.  A design decision was
made to reference objects external to the MAGE-OM and whose primary purpose is
description because doing so constrains the scope of MAGE-OM to describing only
the structure of the microarray experiment and associated quantitative data.
Object metadata is discussed in greater detail in Section \ref{Object
Metadata}.

%XXX

The second area of development in microarray semantics is the \emph{MGED
Ontology}, or \emph{MO}.  The purpose of MO is to allow MAGE-OM objects to be
described in greater detail.  A trade-off was made to use an \emph{ontology},
or a structured controlled vocabulary, in association with the MAGE-OM to keep
the scope of the MAGE-OM itself focused on the description of experimental
processes, 
%XXX

\subsection{Data Storage \& Retrieval}
\label{Storage}

The data and metadata produced as part of pre-processing (Section \ref{Data
Processing}) and fitting the microarray experiment to a uniform data model
(Section \ref{Data Modeling}) must be stored and made available to analysts for
further processing

\subsection{Interoperability \& Exchange Protocols}
\label{Protocol}





\subsection{Northern Blot}
\subsection{Oligonucleotide Synthesis}
\subsection{Robotics}
\subsection{Microfluidics}
\subsection{Photolithography}
\section{Microarray Assay Overview}
A successful microarray experiment is one which is able to provide insight into
the hypothesis for which it was designed.  Insights gained by experimentation
are hard-earned, and require careful planning and execution of multiple key
steps.

The initial steps of feature selection, microarray design, microarray
fabrication, experiment design, and biological sample processing, and image
acquisition are described generally in Sections
\ref{ArrayDesign}-\ref{Hybridization}.  A more exhaustive discussion of
protocol surrounding these steps is given in \cite{wit2004}.  Following their
acquisition, images are quantified, stored, distributed, analyzed, and
analytical results shared.  Challenges and findings related to these latter
procedures are the subject of \emph{\dbthesis}.  They are thoroughly discussed
in Sections \ref{Data Processing}-\ref{Informatics}.

The entire process, from microarray design to data analysis has been formally
described as an object model known as the Microarray Gene Expression Object
Model (MAGE-OM) \cite{mage}.  In this document I have borrowed vocabulary
defined in the MAGE-OM for clarity and consistency.  The MAGE-OM itself is
discussed in Section \ref{MAGE}.

\subsection{Microarray Design}
\label{ArrayDesign}

The first steps in performing a microarray experiment is deciding which genes
to assay on the microarray, how many genes to add, and how to arrange the genes
on the microarray.  These steps are collectively known as \emph{microarray
design}.

Historically, limitations in manufacturing technology were very restrictive,
and limited the number of genes that could be assayed simultaneously.  Current
technology is sufficiently advanced to allow all genes to be simultaneously
assayed for most organisms, so the question of which and how many genes to add
is no longer relevant.  However, as anyone who has performed and experiment
will tell you, the results obtained from the same experimental procedure
performed twice will yield different results.  This also holds true for
measurements made in microarray measurements.  There are many potential sources
of variance in the gene measurements, and a good array design measures as much
of this measurement variance as possible.

Estimation of variance on a microarray essentially means making multiple
measurements of the same gene.  Each gene can be measured at multiple distinct
locations along its sequence.  Further, each location can be measured multiple
times, Further location replicates can be uniformly distributed across the
array surface to compensate for local perturbations on the array surface
itself.

\subsection{Fabrication}
\label{Fabrication}

Once a microarray has been designed (Section \ref{ArrayDesign}) it can be
manufactured.  Fabrication and microarray design are interdependent, in that
improvements in manufacturing techniques allow for the fabrication of more
powerful devices.  Thus, this area of microarray technology is evolving
especially rapidly and any specific details of the synthesis we can describe
here will be outdated before this document is published.  For this reason, I
will only discuss the process of fabrication in general.  A historical
progression of technological advances in fabrication is given in Section
\ref{History}.

\section{Infrastructure}
\label{Infrastructure}

The study of biology is rapidly evolving.  Bottom-up approaches, in which data
are produced individually and slowly are being replaced by high-throughput
methods that produce, in parallel, large volumes of high-dimensional,
high-resolution measurements.  Recent and emerging technologies such as DNA
microarrays, mass spectrometry, and next-generation DNA sequencing enable the
biologist to study the molecular details of entire organisms rather than just a
few genes.  This transition has created new opportunities for discovery in
collaboration with engineers, mathematicians, statisticians, and computer
scientists.  At the same time, this transition necessitates an overhaul of the
computational infrastructure used by biologists.  Specifically, new approaches and
models are required for representing, storing, processing, and distributing
data produced by high-throughput methods.

\subsection{Scalability \& Performance}
\label{Scalability}

One of the key principles that has enabled the creation of high-throughput
methods is scalability.  Simply put, scalable methods are able to process large
amounts of data at the same level of performance as small amounts.  For
biological assays this is largely a problem of parallelization of chemical
reactions and miniaturization of devices.  In terms of computational
infrastructure, building effective, scalable systems also requires
parallelization.  This is often referred to by scalability engineers as
\emph{horizontal scaling} \cite{schlossnagle2006,arlitt2001}.

When setting up an infrastructure system using the \emph{horizontal scaling}
pattern, computer resources can be treated as groups of resources, or
\emph{clusters}.  Each cluster is responsible for one or more types of tasks,
and each component of the cluster is uniform.  This allows the throughput of
the cluster to be scaled simply by adding or removing components.

%See Also
genelogic		\cite{genelogic}

%%%%%%%%%%%%%%%%%%%%%%%%%%%%%%%%%%%%%%%%%%%%%%%%%%%%%%%%%%%%%%%%%%%%%%%%%%%%%%
\section{Informatics}
%@ The role of informatics in biology is dramatically changing the discipline
%and opening up the applications of high-throughput, multiplexing technology
The field of biology is in the process of rapidly changing from a discipline
that largely focuses on bottom up approaches to one that produces and analyzes
vast amounts of high-throughput, high-dimensional data. New emerging
technologies such as DNA microarrays, mass spectrometry, and rapid genome
sequencing mean that a biologist can now study entire organisms rather than just
one or two genes at a time.  This transition has created exciting opportunities
for integrating a wide variety of previously separate research areas including
engineering, statistics, computer science, and other disparate subjects.  The
end result is a field that is swiftly adopting new technologies and approaches
to analyze these highly integrated and challenging datasets. 

%@ Biologists today are facing a huge problem of data sharing, organizing,
%analysing, etc.  Needs solutions to a host of problems.
% Biologists today face huge problems of storing data and organizing large
% datasets such as microarrays.  The typical microarray hybridization experiment
% on the Affymetrix U133A platform contains approximately 22,000 features and
% individual raw image files can take up to 150Mb each of storage space each.
% Pre-processed data files that summarize image files take up less space but
% large datasets can still become difficult to work with.  The Celsius project,
% for example, mirrors public microarray respositories allowing for research into
% comparisons between microarrays from thousands of experiments.  The processed
% expression file for this conglomerate dataset is over 5Tb which is
% prohibatively large for most research labs to duplicate. Furthermore, the
% actual analysis of very large datasets from mass spec or microarray
% technologies can tax computer RAM and CPU capabilities.  The all-by-all
% comparison of probeset correlation, for example, is currently intractable on
% the Celsius dataset on current hardware without first dividing the task into
% smaller steps. For problems where a natural parellel solution is not possible
% limitations of CPU and RAM during analysis become acute issues.

\subsection{Informatic Challenges in Biology}

%@ These core problems break down into several related but distinct activites
% The informatics problems facing biologists can be divided into several distinct
% classes. Issues of working with microarray data can be thought of
% as a microacosum of the common types of problems biologist are experiencing as
% a result of new technologies.  A
% scientist studying gene expression patterns with a large number of microarrays
% experiments must be able to store both the raw and processed results.  These
% might include image files, processed expression data, normalized gene
% expression values, and other derived data.  Experimental annotations must be
% collected that clearly identify samples, how they were treated, and the
% parameters of the physical assay. Next, techniques for analyzing the data must
% be standardized.  This could include common algorithms or best practices for
% working with the data.  In microarrays, many competing analysis techniques are
% available and, regardless of the choice, a common way of accessing them is a key
% concern.  Furthermore, results can change with subsequent releases of new
% software or datasets. Since many scientific questions can take months or years
% to answer, keeping track of what resources were used to produce a given result
% are critical.  Microarray data can be normalized in a number of
% different ways with a variety of algorithms designed with similar but slightly
% different assumptions. Finally, sharing information is the final activity of
% most academic research.  Journals have become more sophisiticated in moving
% articles to completely digital distrubution and archiving models.  However as
% datasets continue to grow access via programatic interfaces is more and more
% important.  This allows for verification of existing results and also the
% synthesis of new questions and research directions.  Microarray data deposition
% in public repositories is a requirement for most journals yet automatic
% mechanisms for working with this data are lacking and requirements for what
% constitues primary data in this field are vague at best. The challenges facin

%@ Commonly applied solutions across the fields of information technology and
%computer science (collectively refered to as informatics) can meet many of
%these needs.  A hybrid approach is needed that blends the best resources from
%a variety of data driven desciplines with the highly parellel nature of modern
%genomics.
The issues facing biologists include storage, annotations, analysis, sharing, and
standardization.  All have roots in the highly technical and interdisciplinary
nature of modern biology.  Bioinformatics, computational biology, and other
related ``new'' fields take an interdisciplinary approach to solving these
challenges.  Commonly applied solutions across the fields of information
technology (IT), computer science, and engineering (collectively referred to
here as ``informatics'') can meet many of these needs.  A hybrid
approach that blends the best resources and practices from a variety
of data driven disciplines with the highly parallel nature of modern biology is needed.
Of particular note, the open source movement has achieved a high level of
success in creating free and flexible software that is philosophically aligned
with the principles of open, transparent, and freely accessible research.  It
is no wonder that most bioinformatics projects use open source software, such
as the Linux operating system, for the bulk of their work.  Solutions for many
of the technical challenges surrounding the transforming field of biology can
be found in the open source movement.

In the following sections, possible solutions to some of the specific informatics
challenges facing biology today are described. Summaries of the open source
bioinformatics projects in this dissertation are available in Table %XXX\ref{projects}.

%@ Informatic solutions borrowed from computer science include...
%@ These solutions can be adapted to the core problems...

%%%%%%%%%%%%%%%%%%%%%%%%%%%%%%%%%%%%%%%%%%%%%%%%%%%%%%%%%%%%%%%%%%%%%%%%%%%%%%
%%%%%%%%%%%%%%%%%%%%%%%%%%%%%%%%%%%%%%%%%%%%%%%%%%%%%%%%%%%%%%%%%%%%%%%%%%%%%%
%%%%%%%%%%%%%%%%%%%%%%%%%%%%%%%%%%%%%%%%%%%%%%%%%%%%%%%%%%%%%%%%%%%%%%%%%%%%%%
%%%%%%%%%%%%%%%%%%%%%%%%%%%%%%%%%%%%%%%%%%%%%%%%%%%%%%%%%%%%%%%%%%%%%%%%%%%%%%


\subsection{Further Work}

XXX XXX XXX

%Chapter \XXXchapter\ provides additional details about the application
%of logic analysis to microarray data.  This includes work supporting the claim
%that logic analysis is able to identify non-random gene expression
%relationships.  Chapter \XXXchapter\ explores the use of logic analysis
%as a general mechanism to discover meaningful gene relationships in additional
%microarray datasets.  It compares the performance of logic analysis to other
%feature selection algorithms and assess the ability of these features selected by
%logic analysis to classify samples in a supervised learning test with a variety
%of classification algorithms.


\section{Placeholder Section}
\label{XXX}

\bibliography{used}
%\bibliography {affy,scholar,pmid,thesis,la,dboutz}    % bibliography references
%\bibliographystyle {uclathes}
\bibliographystyle {plain}

