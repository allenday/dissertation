XXX
\section{Proteomics}

%@ extremophiles definition and our need to understand protein stability
Microbial life has evolved to occupy ecological niches of astounding diversity.
Bacteria and archaea has been found thriving in extremely harsh environments
such as the high-pressure and temperature of deep sea ``black smoker'' vents
\cite{blochl1997pfg}, under brutal extremes of pH ranging between 0 and 11
\cite{2,4}, in the extreme cold of the polar regions \cite{6,7}, and in
high-salt environments such as the Dead Sea \cite{5}.  Together these
extremophiles have changed our perception of how adaptable and resilient life
can be. In doing so, they have provided a window into the mechanisms that
stabilize the molecular machinery of the cell. This has presented a wonderful
opportunity for scientists to gain a deeper understanding of the important
topics of protein stability and adaptation to harsh environments.

\subsection{Thermophiles}

Thermophiles are defined as microbial organisms whose growth temperature
optimum is above 50$^\circ$C while hyperthermophiles have optimal growth
temperatures of 80$^\circ$C or greater. Since Thomas Brock's description of the
hyperthermophilic bacteria \emph{Thermus aquaticus}, with a growth temperature
optimum of 70$^\circ$C, many other thermophiles and hyperthermophiles have been
isolated \cite{brock1969tag}.  Phylogenetic analysis of 16SrRNA indicate that
the thermophiles represent the deepest and shortest branches of the domains
\emph{Archaea} and \emph{Bacteria} \cite{PMID_10376671}.  To date; over 70
thermophilic archaea have been described while relatively few thermophilic
bacteria have been cultivated, most belonging to the orders
\emph{Thermotogales} and \emph{Aquificiales} \cite{PMID_11256505}. Model
organisms for archaea belong to the genera \emph{Pyrococcus}, while bacteria
come from the genera \emph{Thermotoga}. Sequencing efforts have made
available the genomes of over three dozen archaeal and approximately 400
bacterial genomes including thermophiles, mesophiles, (organisms growing in
standard environmental ranges), halophiles (salt-adapted), acidophiles (low
pH-adapted), alkaliphiles (high pH-adapted), and barophiles (pressure-adapted).
The availability of these genomic sequences from the National Center for
Biotechnology Information (NCBI, \url{http://www.ncbi.nlm.nih.gov}), coupled
with many high-resolution protein structures from the Protein Data Bank (PDB,
\url{http://www.rcsb.org}), has opened up the field of thermophile research to
proteomic and genomic studies previously impossible.  

% NOTE: phylogenetic analyses of 16SrRNA indicate hyperthermophiles represent
% the deepest and shortest branches of teh domains Archaea and Bacteria
% (Sterner_2001_ref_setter_1999).

%@ How stability is quantified Gibbs free energy change can calculate the
%change in energe from the denatured to the native form of the protein
%structure.  An increase in delta G is associated with an increase in protein
%stability 
% These rich datasets have already provided an understanding of the specific
%mechanisms by which proteins can be stabilized in these harsh environments.
Armed with this extensive collection of data, many studies have looked at the
specific biochemical nature of protein stability in thermophiles
\cite{jaenicke1998spe,PMID_11577980,PMID_10097079,PMID_10940293,thompson1999tel,PMID_11256505}.
Much attention has been paid to thermophiles in particular because an
understanding of the mechanisms governing protein stability in these organisms
are applicable to the protein stability question in general.  Given the
observation that thermophilic proteins expressed in mesophiles continue to
exhibit greatly increased thermostability, the increased stability must be a
product of the primary amino acid sequence \cite{razvi2006lst}.  Understanding
the sequence criteria that enable protein stability under such high thermal
stress should be possible by examining thermophilic protein sequences and
structures.  Much of this section of the introduction reviews the findings
associated with these datasets.

\subsection{Thermodynamics and Protein Stability}

A measure of protein stability is required to understand and quantify the
effects of various stabilizing strategies utilized in thermophilic proteins.
The common measure of protein stability is the change in Gibbs free energy
associated with the transition from the protein's native fold (N) to a
denatured (D) polypeptide:  

\begin{equation}\label{equ}
  %D \xrightleftharpoons[\text{${k}_{d}$}]{\text{${k}_{n}$}} N
  N \rightleftharpoons D
\end{equation}

% dG neg: reaction spontaneous N -> D
% dG = 0: forward and reverse occurr at eq rates
% dG pos: the reverse reaction D -> N
% so the dG(T) curves are talking about the reaction N -> D
% because you want to see high pos dG indicating the reaction from
% N -> D is not favored
% dS is positive and S is positive and given in kJ/K (so when times by T you get kJ)
% dH is positive and H is negative and given in kJ
The change in Gibbs free energy, or $\Delta$G, for the transition from denatured
to native states is calculated with the following equation:

\begin{equation}\label{gibbs}
  \Delta G = \Delta H - T \Delta S
\end{equation}

% delta H = H final - H initial, exothermic is -, endothermic +
$\Delta$H in equation \eqref{gibbs} refers to the change in enthalpy, or heat
content; T refers to the temperature; and $\Delta$S refers to the entropy in
the transition from the folded native state of the protein (N) to an unfolded state
(D).  Enthalpy is a measure of the various favorable interactions within a
given state.  The enthalpy of the transition from the N to D-states ($\Delta$H)
can be thought of as the amount of energy necessary to disrupt favorable
interactions in the native state minus the energy of new interactions
gained in the disordered state.

% \begin{equation}\label{dH}
%   \Delta H = \Delta {H}_{denatured} - \Delta {H}_{native}
% \end{equation}

The change in entropy ($\Delta$S) can be thought of as the difference in disorder between
the native state of the protein and its denatured state. In the denatured state
the entropy of the protein is highest because it has the most conformational
flexibility and freedom.  In the transition to the native state the possible
conformations of the protein are greatly constrained. The change in entropy can
be thought of as the energy lost to the order established in the transition
from a disordered to ordered protein.

% \begin{equation}\label{dS}
%   \Delta S = \Delta {S}_{denatured} - \Delta {S}_{native}
% \end{equation}

% An increase in a protein's stability is seen by an increase in $\Delta$G.
% If a protein's stability is increased in the transition from disordered to
% native state than the $\Delta$G is increased as well. 
A protein can increase $\Delta$G, and thereby improve its stability, by 
increasing the enthalpy of the native state \cite{119}, decreasing the enthalpy or
entropy of the disordered state \cite{120,121}, or both \cite{122,123}. 

%@ T of melting is the general metric for measuring stability. Info from paper
%that talks about stategies for changing these variables. This will just talk
%about general stategies not which thermophiles focus on.
In order to understand how a proteins stability changes as a function of
temperature, Becktel and Schellman \cite{becktel1987psc} introduced protein
stability curves.  These are a plot of $\Delta$G as a function of temperature,
and include ${T}_{m}$, the temperature of melting when $\Delta$G equals zero
and the protein is at the midpoint of transition from native to its disordered
state. ${T}_{m}$ is often used as an indicator of protein stability.  These
curves are described by a modified version of the Gibbs-Helmholtz equation:

\begin{equation}\label{GH}
%\begin{displaymath}
 \Delta G \left (  T \right )  =  \Delta  {H }_{m }  \left (  1 -  \frac{ T }{  {T }_{m } }  \right ) -  \Delta  {C }_{p } \left[   \left (  {T }_{m }  - T \right ) + T \ln  \left (   \frac{ T}{ { T}_{ m}  }  \right )  \right ] 
%\end{displaymath}
\end{equation}

In equation \eqref{GH} $\Delta G \left (  T \right )$ is the free energy at
temperature \emph{T}, $\Delta  {H }_{m }$ is the change in enthalpy at ${T }_{m
}$, and $\Delta  {C }_{p }$ is the change in heat capacity associated with the
denaturing of the protein. Heat capacity refers to the amount of heat a protein
can store at a given temperature.  Figure \ref{Sterner_Fig_1} shows a
hypothetical protein stability curve with key thermodynamic parameters noted.

% sterner_ref_pace_2007 to pace1997mcs
In order to find $\Delta$G as a function of temperature three variables must be
determined experimentally: the temperature of melting (${T }_{m}$), enthalpy
change ($\Delta$H), and heat capacity change ($\Delta{C}_{p}$)
\cite{pace1997mcs}. ${T}_{m}$ can be found with differential scanning
calorimetry (DSC) or circular dichroism (CD) spectroscopy by following the
degree of unfolding upon heating. A value for $\Delta$H only needs to be
determined at one temperature; typically this is $\Delta{H}_{m}$ which is the
enthalpy change at ${T}_{m}$. DSC or CD spectroscopy can be used to measure
the heat of unfolding ($\Delta$H) and $\Delta{C}_{p}$ can be calculated given the
relationship to enthalpy:

\begin{equation}\label{cp}
\frac{ d \left ( \Delta H \right ) }{ dT }  = \Delta {C}_{p}
\end{equation}

Using equation \eqref{cp}, $\Delta{H}_{m}$ and ${T}_{m}$ can be measured at a
variety of pH values using DSC.  The resulting variation in ${T}_{m}$ and the
corresponding variation in $\Delta{H}_{m}$ can be plotted and a value for
$\Delta{C}_{p}$ determined from the slope of the line \cite{privalov1979sps}.
$\Delta  {C }_{p }$ is an important variable since it determines the curvature
of the $\Delta$G graph.  The smaller the $\Delta{C}_{p}$, the flatter the curve
and, consequently, the higher the ${T}_{m}$. Once $\Delta{C}_{p}$,
$\Delta{H}_{m}$, and ${T}_{m}$ are determined a plot of $\Delta G \left (  T
\right )$ can be made and compared to other proteins.

% FIXME: transition to next paragraph needs work.

% NOTE: 
% * enthalpy (H): heat content, description of the thermodynamic potential of a
% system. If it's neg then the reaction is exothermic, if its positive then the
% reaction is endothermic. 
% * entropy (S): spontaneous changes occur with an increase in entropy of
% disorder
% * heat capacity (Cp): measure of ability of a body to store heat as it
% changes in temperature. unfolded has higher heat capacity than folded.
% increases linearly w/ amino acids so less amino acids = smaller Cp = higher
% Tm 

% NOTE: Sterner pg 44: enthalpy change (delta H) and heat capacity (delta Cp)
% between the denatured and native states must be determined in order to find
% delta G as a function of Temp.  To find delta Cp you use d(delta H)/dT =
% delta Cp, so delta Cp is given by the derivative of delta H w/ respect to
% time.  To find this you varry the pH of the solution you're measuring and
% this varies Tm and delta H which can be followed by differential scanning
% calorimetry (DSC).  Take the slope of the line and this is delta Cp (delta H
% increases with temperature since the unfolded state has a higher heat
% capacity than the folded state so Cp is pos).  Delta H can be measured
% directly with CD, usually at delta Hm, the enthalpy at Tm.

% NOTE: Figure 1 on Sterner pg 45 would be great to include.

% NOTE: Sterner pg 43. Three points need adaptation to survive extreme temps:
% membranes, DNA, and protein stability.  Cell membrane tetraether lipid
% monolayers found in extremely thermophilic archaea are more stable than
% ester-type lipid bilayers.  Ether links are more resistent to temp than ester
% \cite{sterner_ref_van_de_Vossenberg_1998}.

% NOTE: Thermo damage: deamidation, beta-elimination, hydrolysis, Maillard
% reactions, oxidation, disulfide interchange (Sterner pg 43)

% NOTE: pg 50 Sterner, talks about the various methods seen by Jaenicke 1998
% and also make the important point that these effects can 1) be interrelated
% w/ each other and 2) vary based on heat... include his examples here.

%@ There are several different methods by which a protein can increase its Tm,
%temperature of melting and achieve higher thermostability. The specifics in
%terms of how do you go from changes to delta G to actual changes in proteins

% FIXME: Add this in somewhere: Razvi \emph{et al.} described several ways in
% which the thermodynamic qualities of a protein could be modified to increase
% ${T }_{m }$ \cite{razvi2006lst}.  

\subsection{Interpretation of Thermodynamic Models}

Organisms adept at surviving hostile, high-heat environments necessitate
specific and widespread changes to their proteins.  Unlike lipids, whose
composition can be changed depending on the needs imposed by the environment,
proteins are limited to a restricted set of amino acids
\cite{Razvi_ref_Russell_1990}.  Adaptations to environmental pressures must be
accomplished through changes to protein sequence. Nojima \emph{et al.} proposed
three different methods for changing protein stability curves ($\Delta G \left
(  T \right )$) and, thus, changing ${T}_{m}$ \cite{Razvi_Nojima_1977}.  These
were: shifting the overall curve up, broadening the curve, or shifting the
curve to right.  Figure \ref{Sterner_Fig_1} illustrates these three techniques 
and all three have been seen at varying degrees in thermophilic proteins
with the result of shifting the ${T }_{m }$ to a higher temperature.  

The various mechanisms were probed in Razvi \emph{et al.} with a survey of 26
protein families including examples from mesophiles and thermophiles
\cite{razvi2006lst}.  They found the most common way to attain a higher ${T
}_{m }$ is to raise the overall $\Delta G$ stability curve and this was
utilized by 77\% of the proteins they examined.  This is consistent with the
general observations that thermophilic proteins have increased salt-bridges,
hydrogen bonds, and hydrophobic interactions all of which can directly increase
$\Delta G$ through increasing the enthalpic interactions of the native state.
They also observed the widespread lowering of $\Delta  {C }_{p }$ in 70\% of
the proteins studied effectively increasing ${T }_{m }$ as well.  $\Delta {C
}_{p }$ has been shown to increase linearly with protein size and is increased
in the unfolded state \cite{myers1995dmv}. This is consistent with the
observation that thermophiles generally have smaller proteins than mesophiles
as judged by a variety of methods \cite{chakravarty2002efr}.  In addition, this
change in heat capacity is lowered by increasing the order of the denatured
state, by adding structured components that persist in this state, and by
providing for tighter core packing in the native state.

% FIXME: Now add some transition saying these changes to thermodynamic
% qualities of the protein are the result of changes to the specific coding
% sequence of the protein.  Each specific strategy for latering key
% thermodynamic quatilies and parameters is the result of one or more
% strategies being played out in the sequence of the genome.

% NOTE: in Sterner pg 50 there are a ton more methods mentioned.  I think I
% should cover them then reference the most recent work (Razvi?, Kumar gives
% the short list, they both look at protein families but Kumar gives more
% specifics and Razvi is more about the thermodynamic changes to proteins).
% jaenicke1998spe,PMID_11577980,PMID_10097079,PMID_10940293,thompson1999tel,PMID_11256505

These specific examples of adaptations to increase ${T}_{m}$ are only a few of
the many ways proteins can be altered to increase their overall stability.
Studies have examined collections of mesophile and thermophile proteins in
order to understand the specific mechanisms related to the shifts measured by
the thermodynamic analysis detailed above \cite{jaenicke1998spe}.  These
mechanism are many of the same mechanisms used in mesophilic proteins but there
tend to be more per protein.  For example, thermophilic proteins exhibit (1)
more hydrogen bonds \cite{PMID_11577980}, (2) additional electrostatic
interactions such as increased numbers of salt bridges \cite{beeby_31,PMID_10097079},
(3) tighter packing of the hydrophobic protein core \cite{beeby_29}, (4)
optimized hydrophobic interactions, (5) increased polar surface areas
\cite{PMID_10940293}, (6) increased and more stable $\alpha$-helices
\cite{PMID_11577980}, (7) improved metal ion binding, (8) improved integration
of the polypeptide chain termini, (9) replacement of amino acids in unfavorable
conformations with Gly, (10) shorter disordered loops \cite{thompson1999tel},
(11) more prolines in unstructured loops, (12) increased oligomeric
associations, and (13) the reduction of thermolabile amino acids
\cite{PMID_11577980,PMID_10097079}. Figure \ref{Sterner_Table_1} shows the
comparison by Haney \emph{et al.} of amino acid usage patterns in a thermophile
(\emph{Methanococcus janaschii}) and several related mesophilic
\emph{Methanococcus} species \cite{PMID_10097079}.  The trend to increase electrostatic interactions
can clearly be seen with Arg, Lys, Glu being more common in the thermophile.
The opposite effect can be seen with Ser, Met, Asn, and Gln, which are
thermolabile amino acids and, consequently, depleted in the thermophile.
% Figure \ref{Sterner_Table_2} highlights some of the structural differences
% between mesophiles, thermophiles, and hyperthermophiles.

% FIXME: maybe remove Fig sterner_table_2?

% last refs above: beeby_32-37

% NOTE: rather than go into details here I may want to introduce the
% thermodynamic framework first then offer concrete examples found by examining
% protein families.  Other things to mention there, hydrophobic interactions
% and electrostatic interactions increase w/ increasing temperature.  This sort
% of info is sparse (see Sterner pg 50 for refs).  In a study by Haney 1999 in
% Methanococcus janaschii vs. mesophile Methanococcus.  Decrease in uncharged
% polar residues Gln, Asn, Thr, Ser (net loss of 9 in 300aa protein), Gln and
% Asn prone to deamidation catalyzed by Ser and Thr residues (Wright 1991).
% Increase in charged amino acids Glu, Arg, Lys (net gain of 6.5 residues in a
% typical protein.  Trends applied to 88% of the proteins in the study.
% Cambillau 2000 confirmed results, looked at 22 mesophiles and seven
% thermophiles. Results suggest extra charged residues are located at the
% surface region of the hyperthermophilic proteins. Chakrravarty 2000 average
% thermophilic protein contains 268+- 38 amino acids vs. mesophilic proteins
% that contain 310+-16 residues.  Also Thompson 1999 thermophilic proteins
% shorter, due to shorter solvent exposed loops.  So Cp decreased by smaller
% proteins and Entropy lowered in unfolded state.  (doesn't that increase the
% deltaS term which decreases deltaG? This is a little unclear for me.) Loop
% shortening would have a multiplicative effect on the number of confirmations
% for the entire chain in the unfolded state (does this mean greater entropy in
% the unfolded state?). 

% NOTE: Systematic Structure Comps: Szilagyi 2000 looked at 29 thermo and 64
% meso protein structures. Found 1) no difference in normalized number of
% hydrogen bonds, total surface areas of cavities decreased in thermophiles
% (s100), ion pairs were the greatest diff between meso and thermo.  300 aa
% protein would see 4 strong and 14 weak additional ion pairs at 80C vs meso.

% NOTE: Kumar 2000 had a similar study where they looked at 18 protein familes
% (I have this paper).  Looked at packing densisty, hydrophobicity,
% polar/non-polar surface areas, protein size, Pro in loops, helical content,
% helical stability, hydrogen bonds, ion pairs, and amino acid composition.
% Found the following are different: thermophiles more alpha helicies, more
% stable (less His, Cys, and Pro in them and more Arg), thermolabile aa (Cys
% and Ser) reduced while Arg and Tyr more, more salt-bridges and
% side-chain/side-chain hydrogen bonds.

% NOTE: Karshikoff 1998 (Sterner pg 56) looked at packing and found no
% significant diff

% NOTE: Facchiano 1998 alpha-helicies more stable due to amino acid choices
% (Sterner pg 57).

% NOTE: Sterner summarizes the most common stabilization methods on page 58.
Despite years of research, no clear and decisive mechanism for
thermostability has come to light.  Instead, thermophiles appear to use a
variety of these mechanisms to stabilize their proteins. It is believed that the
summation of these weak interactions is the source of thermophilic protein
stability and the contribution of each mechanism can vary from protein to
protein
\cite{chakravarty2002efr,PMID_11577980,vieille2001hes,jaenicke1998spe,petsko2001sbt,rees1995hth}.
Sterner and Liebl summarized the most consistent mechanisms used in a wide
variety of protein families from thermophiles as based on many different protein
structure studies
\cite{PMID_10097079,PMID_11577980,beeby_31,beeby_29,PMID_10940293,thompson1999tel}.
The most commonly observed adaptations include: (1) the stabilization of
$\alpha$-helices, (2) an increase in the percent of proline residues resulting
in reductions in the entropy of unfolded proteins, (3) a decrease in
thermolabile amino acids (Cys and Ser), and (4) an increase in the number of
electrostatic interactions and their optimization, such as increased numbers of
salt bridges \cite{PMID_11256505}. These four general mechanisms are the most
conserved adaptations seen across the thermophilic proteins analyzed to date.
A role for structurally stabilizing disulfide bonds, however, was not a
documented strategy.

\subsection{Disulfide Bonds \emph{in vivo}}

% deltaG 5-15 kcal/mol for globular protein which covalent bond is 30-100 kcal/mol
% http://public-1.cryst.bbk.ac.uk/PPS2/projects/day/TDayDiss/StabilityDefined.html

%@ Disulfide bonds are typically seen in proteins fullfilling two distinct
%roles, structural or catalytic. For structural roles they effect the
%enthalpy...
% refs: boutz_35-38 to 32,33,34,35
% Boutz_ref_118-120 to 119,121,120
Disulfide bonds are typically seen in prokaryotes and eukaryotes as part of 
active sites in cytosolic proteins or as structural components that stabilize
proteins in the extracellular environment.  In cytosolic proteins, the
formation of a disulfide happens during catalytic reactions and the cell
generally maintains the active site cysteines in a reduced state
\cite{32,33,34,35}.   Disulfides in extracellular proteins increase the
enthalpy in the native state while decreasing the conformational freedom, and,
effectively, decreasing the entropy of the denatured state.  Which effect is
more pronounced in the stabilization of a disulfide-containing protein varies
\cite{119,121,120}. Mutation studies in lysozyme, for example, have identified
specific changes that can alter the ${T }_{m }$ either primarily through
changes in $\Delta H$ or $\Delta S$ \cite{125,124}. 

\subsection{Disulfide Bonds in Prokaryotes}

%@ The generally reducing environment within cells makes the regulation of
%disulfides complex.  In E. coli disulfides are maintained through a complex
%pathways of redox proteins... Thioredoxin pathway...
% boutz_ref_33 to 30
% boutz_ref21 to 19
% boutz_34 to 31
\emph{In vivo} studies in \emph{E. coli} and yeast have revealed that disulfide
bonds form almost exclusively in extracellular and compartmentalized proteins
and this was attributed to the generally reducing nature of the cytosolic 
environment. Glutathione acts as the general redox buffer in \emph{E.  coli} and
other Gram-negative bacteria and eukaryotes. It is present at $\sim$5mM
concentration with an electrochemical potential (E$^\circ$) of approximately
-240mV \cite{30}.  The ratio of reduced to oxidized glutathione is
approximately 200:1 in favor of the reduced form \cite{19}.  The balance of
reduced and oxidized forms of glutathione in the cell are currently understood
to be used to buffer oxidative stress \cite{31}.  
% were once thought to be the primary reductant of cytosolic proteins however
% it is

% boutz_ref_12-22 to 11,144,12,13,14,15,16,17,18,19,20
Disulfide bond formation in the cytosol of \emph{E. coli} and yeast is
clearly discouraged given the extremely reductive nature of the environment.
It was originally thought that small molecule oxidants, such as oxygen, were
responsible for transient disulfide formation and reduced glutathione was
thought to reduce these cytosolic proteins.  Over time, though, a new role for
enzymes as controllers of the redox reactions began to emerge
\cite{11,144,12,13,14,15,16,17,18,19,20}.  Studies began to link complex
biochemical enzymatic pathways in \emph{E. coli} with protein disulfide
regulation.  These pathways regulate the formation of disulfide bonds in
extracellular and, in the case of eukaryotes, organelle environments while
ensuring the cytosolic proteins remain reduced.  

% FIXME: Check the last sentance

% boutz_27-30 to 25,carvalho2005sad,26,27
% boutz_45 to 41
There are two primary enzyme families that maintain the reduced state of
cytosolic proteins: the thioredoxins and the glutaredoxins.  The thioredoxins
contain a characteristic fold and a common CXXC motif, where X refers to any
amino acid \cite{25,carvalho2005sad,26,27}.  In prokaryotes, such as \emph{E.
coli}, the thioredoxin pathway consists of two enzymes families: thioredoxin
(Trx1 and Trx2) and thioredoxin reductase (TrxR). Trx1 has a redox potential of
around -270 mV making it the enzyme with the strongest reducing power
\cite{11}. This is the primary pathway responsible for the reduction of
intramolecular disulfides in the cytosol.  In the process, thioredoxin becomes
oxidized and is recycled by thioredoxin reductase which uses NADPH for its
reducing activity \cite{41}.  This couples the reducing potential of the
thioredoxin pathway to respiration.

%@ The glutaredoxin pathway...
% boutz_19_49_50 to 47,48,17
The glutaredoxin pathway is in many ways parallel to the thioredoxin pathway
and consists of glutaredoxin (Grx1, Grx2, and Grx3), glutathione, and
glutathione reductase.  In this system the maintenance of glutathione in the
reduced form is due to the action of glutathione reductase.  Glutathione
reductase converts oxidized glutathione to reduced glutathione using the
reductive power of NADPH.  The reduced glutathione can then go on to reduce
glutaredoxin which, itself, targets other proteins for reduction.  The specific
overlap between the pathways of thioredoxins and glutaredoxins is yet to be
fully understood.  Generally, glutaredoxins are less efficient at reducing
protein disulfide bonds \cite{47,48,17}. 

% FIXME: check the last sentance.

%@ the pathway for periplasmic disulfide bonds in E. coli...
The prokaryotic cell is able to maintain reduced disulfides in cytosolic
proteins using these two pathways.  However, the periplasmic space and
extracellular environment are rich with disulfide-containing proteins.  As with
the cytsolic proteins, the regulation of disulfides is carefully controlled by
enzymes that oxidize and isomerize cysteine pairs. This control in \emph{E.
coli} is the result of redox-active enzymes from the Dsb protein family. DsbA
acts as the oxidizer of cysteine pairs via a CPHC
active site motif and, with a redox potential of $\sim$-120mV, is the strongest
thiol-oxidizing enzyme identified thus far \cite{59}. Once reduced, DsbA must be
re-oxidized by DsbB, a transmembrane protein that shuttles
electrons from the reduced form to the electron transport chain via ubiquinone
or menaquione under anerobic conditions \cite{87,88}.

% boutz_68_69 to 67,68
% boutz_59_75 to 54,59,61,62,58,63,64,66,65,67,68,70,71,72,73,69,78
% FIXME: I think it's 59 and 75 not 59-75!!!!
Oxidation of the substrate protein by DsbA is non-specific and the correct
disulfide bond formation is catalyzed by the DsbC disulfide isomerase protein.
DsbC contains a thioredoxin-like fold with a CXXC motif and exists as a
homodimer. In addition to the isomerization activity, DsbC also seems to act as
a protein folding chaperon \cite{67,68}.  This activity is linked with
disulfide isomerization because proteins with incorrectly resolved disulfides
are also likely to be misfolded.  During the isomerization process the active
site cysteines in DsbC become oxidized, necessitating reduction by DsbD
\cite{54,78}.  This transmembrane protein includes three domains: DsbD$\alpha$,
DsbD$\beta$, and DsbD$\gamma$.  These domains each contain an active site
cysteine and together, shuttle electrons from Trx1 and donate them to DsbC
\cite{64,79}. DsbD also reduces DsbE and DsbG.

A summary of the disulfide pathways in \emph{E. coli} is presented in Figure
\ref{Boutz_Fig_3}.

\subsection{Disulfide Bonds in Eukaryotes}

% boutz_95 to 95
% boutz_102,103 to 102,103
% boutz_105_107 to 105,106
%@ In eukaryotes PDI catalyzes...  Ero1p shuttles electrons to oxygen from PDI
In eukaryotes, the pathway for disulfide reduction and oxidation shares similar
underpinnings yet the actual proteins are different.  The glutathione system is
maintained with a reduced to oxidized ratio of $\sim$100:1 and non-catalytic
disulfides are extremely rare in the cytosol \cite{95}.  However, the lumen of
the ER does host proteins that are destined for export from the cell and often
contain disulfides.  The ratio of reduced to oxidized glutathione in this
compartment is $\sim$3:1 making the redox environment much more conducive to
disulfide formation \cite{95}.  Protein disulfide isomerase (PDI) acts as both
an isomerase and a oxidizer in this compartment \cite{96,99}.  It exists as a
monomer and contains four thioredoxin-like domains and two CXXC motifs.  This
protein has also been shown to possess a chaperon role and assist in protein
folding as well \cite{97}.  This functionality suggests a role in the cell
similar to a combination of DsbA and DsbC.  PDI is maintained in an oxidized
form by the membrane-associated flavoprotein Ero1p \cite{102,103} which
shuttles electrons from PDI to molecular oxygen \cite{105,108}.  As with the
redox system in \emph{E. coli} and other prokaryotes, the disulfide bond
maintenance pathway is complex and specifically controlled.  Yet much remains
to be discovered about the regulation of this import process.

\subsection{Disulfide Bonds in Intracellular Proteins}

%@ Given the general techniques for protein stabilization and the specific
%utilization of disulfides there are still questions about which methods
%thermophiles use... They use some general techniques for stabilization just as
%mesophilic proteins do.  Previous results lead to an interesting hypothesis
%about a more ubiquitous use of disulfides in thermophiles... Original evidence
%was ACL... then looked at disulfide structure prediction...
Disulfide bonds are clearly an important component driving protein stability in
the extracytoplasmic environment for both prokaryotes and eukaryotes.  Despite
their potential utility in stabilizing proteins, the common doctrine excludes
them from a role within the cytosol of all organisms.  As with \emph{E. coli}
and yeast, it was assumed the reductive nature of the cytosol renders disulfide
bonds extremely rare.  A role for disulfides in intracellular proteins
from thermophiles was, as a consequence, ignored.  Furthermore, proteins of
archaea are generally low in cysteine content and this was seen to be a direct
consequence of the thermal labile nature of cysteine \cite{PMID_10097079}. The
high temperatures associated with thermophilic growth were predicted to cause
oxidative degradation of the cysteines.

% mallick_12 to toth2000sal
% mallick_14-18 to jiang1996sai,maes1999cst,dedecker1996csh,chiu2001csn,meyer2002hpt
Recent computational work and protein structures have altered the
perception of structural disulfides in cytosolic environments and opened up
the possibility that they could be commonplace in thermophiles
\cite{beeby2005gdb,mallick2002gei}.  Evidence for potentially wide-spread
disulfide bonds in thermophiles came from the surprising discovery of three
disulfide bonds in the protein structure of adenylosuccinate lyase (ASL) from
\emph{Pyrobaculum aerophilum} \cite{toth2000sal}. This was shocking since the
protein is cytosolic. This was not the first protein structure from a thermophile that
contained disulfide bonds, but ASL prompted additional scrutiny via
computational studies
\cite{jiang1996sai,maes1999cst,dedecker1996csh,chiu2001csn,meyer2002hpt}.  

\subsection{Computational Prediction}

The first study by Mallick \emph{et al.} directly examined the possibility that
disulfides were actually common in intracellular proteins of thermophiles
\cite{mallick2002gei}.  Protein sequences from 25 microbial species, including
the thermophile sequences available at the time, were queried against the PDB.
Proteins where then threaded onto the backbone of their best match in the PDB
and the distances between cysteines were assessed.  Efforts were taken to
remove transmembrane proteins, those with thioredoxin folds and CXXC motifs,
extracellular proteins, and proteins with other metal-binding signatures.  The
results were shocking and indicated an enrichment for disulfide bonded
cysteines in cytosolic proteins for the thermophiles examined.  Nine of the
genomes analyzed were predicted to contain $>$10\% of their cysteins in
disulfide bonds.  \emph{P. aerophilum} and \emph{A.  pernix} were predicted to
have 44\% and 40\% of their cysteines in disulfide bonds respectively.
Fluorescent labeling experiments confirmed an enrichment of disulfide bonds in
lysate from \emph{P.  aerophilum} versus \emph{E. coli}.

% FIXME: use this????  Landenstein_ref_8 to karshikoff1994ept This asstonishing
% result opened up the possibility that disulfide bonds are common in the
% cytoplasmic environment of many thermophiles and these covalent bonds may
% play a key stabilizing role in thermophilic proteins.  Many studies have
% examined the role of various techniques in stabilizing thermophilic proteins
% with the mechansism used to increase $\Delta$G and ${T}_{m}$ discussed above.
% An examination of electrostatic interactions may point to a key reason
% disulfides are used in these organisms \cite{Landenstein_ref_8}.  This work
% showed that low levels of spatial optimization of the electrostatic
% interactions within a protein are compensated by covalent cross-links and
% vice versa. Smaller proteins tend to have a lower electrostat optimization
% values and this agrees with a higher disulfide-densisty for the generally
% smaller proteins of the thermophiles \cite{ladenstein2006pda}.

% Landenstein_10 to karlstrom2005idh
% Landenstein_15 to bjork2004lit
% Landenstein_10 to karlstrom2005idh
The role of disulfides as stabilizing components of thermophilic proteins has
been supported by work with two cytosolic proteins from \emph{P. aerophilum}
and \emph{A. pernix}.  Isocitrate dehydrogenase from \emph{A. pernix} was
mutated (Cys87Ser) resulting in the disruption of a disulfide bond.  The
resulting melting temperature difference of -9.6 $^\circ$C points to a direct
role for this covalent bond in stabilizing the protein \cite{karlstrom2005idh}.
Another example includes the disruption of disulfide bonds with dithiothreitol
in ASL from \emph{P.  aerophilum}, resulting in a shift of the ${T }_{m }$ by
-18.5 $^\circ$C \cite{toth2000sal}.  Other examples follow the same pattern
with disulfides in thermophilic proteins providing stabilization and increased
${T }_{m }$ \cite{ladenstein2006pda}.  Similarly, engineered disulfides in
malate dehydrogenase from \emph{Chloroflexus aurantiacus} induced an increase
in ${T }_{m }$ of 15$^\circ$C \cite{bjork2004lit}.

\subsection{A Novel Disulfide Oxidoreductase}

%@ Interestingly, disulfide richness is not ubiquitous in thermophiles, sulfur
%reducers, for example, are lacking PDO and seem to have few disulfides.
% beeby_32-37 to chakravarty2002efr,PMID_11577980,vieille2001hes,jaenicke1998spe,petsko2001sbt,rees1995hth
Continued computation research pointed to a more nuanced view of disulfide bond
utilization in cytosolic proteins from thermophiles.  This is the topic of Chapter
\disulfidechapter\ but briefly, the original computation study was scaled to
assess the disulfide bond abundance across almost 200 microbial species.  This
work supported the original assessment that disulfide bonds are widespread in
the genomes of many, but not all, thermophiles. This corroborates the idea that
many different techniques are used to stabilize proteins and one single method
does not dominate all thermophiles
\cite{chakravarty2002efr,PMID_11577980,vieille2001hes,jaenicke1998spe,petsko2001sbt,rees1995hth}.
The thermophiles predicted to have few disulfide bonds include the methanogens,
many sulfur-reducing organisms, and the thermophilic cyanobacteria.  Many of
these organisms grow in extremely low redox environments under strongly
reducing conditions and this biological niche may preclude the use of
widespread disulfide bonds.  In addition to thermophiles other extremophiles
appeared to include enrichments for disulfide bonds.  These included some
halophiles, alkaliphiles, acidophiles, and radiation tolerant microbes
suggesting that disulfide bonds may be an even more widespread technique to
stabilize proteins in extreme environments \cite{beeby2005gdb}.

%@ Mechanisms to enable disulfide bond formation in the cytosol are still
%unknown yet PDO was found to correspond very distinctly with thermophilicity
%and disulfide prediction leading to a possible involvement... further crystal
%structures showed...  Since the possible mechansism is not understood, redox
%buffer is not even known in these species, a true role for PDO cannot be
%assigned.
% ladenstein_ref_5 to darby1995cmd
This tantalizing possibility points to a radically different biochemical
environment within the cytosol of these poorly understood organisms.  New redox
enzymes are being discovered that are not members of the thioredoxin
superfamily but serve redox roles in the cell.  These use electron donors or
receptors such as FAD, NADPH, NADH, quinones, and lipoic acid
\cite{darby1995cmd}.  The specific pattern of the enrichment of disulfide bonds
in thermophiles was studied and related to protein phylogenetic profiles as is
detailed in Chapter \disulfidechapter.  This allowed for the precise correlation of
protein disulfide isomerase (PDO) with the quality of richness in disulfide
bonds \cite{beeby2005gdb}.  Given that the presence of PDO in these organisms is not
found only on a single branch of the tree of life, the specificity to which it
correlates with disulfide richness is strong evidence for its involvement with
the quality \cite{ladenstein2006pda}. 

% landenstein_ref_20 to guagliardi1994tet
% landenstein_45,40 to pigiet1986tcr,bardwell1991ipr
The PDO protein contains a C-terminal region of similarity to glutaredoxin, two
general thioredoxin-like folds and includes four cysteines into two CXXC motifs
in its sequence. X-ray structures for both the \emph{Pyrococcus furiosus} and
\emph{Aquifex aeolicus} proteins have been solved.   The protein from
\emph{Sulfalobus solfataricus} was shown to reduce insulin disulfides
indicating a catalytic potential \cite{guagliardi1994tet}.  Yet the actual
\emph{in vivo} nature of PDO and its role in disulfide bond formation is
elusive.  The ability of PDO, like all other oxidoreductases, can vary based on
the cellular environment \cite{pigiet1986tcr,bardwell1991ipr} and its ultimate
place in what is likely to be a complex pathway for disulfide bond formation
and maintenance has yet to be seen.  A detailed understanding of disulfides in
these unusual organisms will likely include many proteins and emerge as more
genomes are sequenced and genetic systems in thermophiles become available.

\subsection{Further Work}

The discovery of the nuanced distribution of disulfide across microbial species
and the identification of PDO is covered in detail in Chapter \disulfidechapter\ of this
dissertation.  Chapter \gdapchapter\ introduces a web tool for predicting disulfides in new
sequences. 




\newpage

\section{Genomics}

%@ Gene expression studies look at... The type of questions they attempt to
%answer...

DNA microarray technologies are a recent advancement in biological research
that have found widespread adoption among a diverse set of research subdomains.
They represent one of the most common high-throughput experimental approaches
biologist now have at their disposal to answer questions that were previously
intractable. Microarrays provide the ability to simultaneously measure the mRNA
abundance of thousands of genes in a given biological sample and do so
relatively quickly and cheaply.  The power of this approach is evident in the
large number of diverse studies that have been performed.  These experiments
can be grouped into one of several general categories based on the overall
study design and end goal.  These include: classification problems (supervised
learning) \cite{Dudoit2003ICM}, clustering problems (unsupervised learning)
\cite{Azuaje2003cge,Stanford2003cac}, gene co-regulation and function
identification studies \cite{PMID_12413821}, differential gene expression
studies \cite{PMID_15843092,PMID_15867208}, time-course and dose-response
studies \cite{PMID_12443997}, gene pathway and regulatory network studies
\cite{PMID_16216773,PMID_16825123}, drug discovery and toxicology studies
\cite{PMID_16512775,PMID_16700885,PMID_16880944,PMID_17195470}, clinical
diagnostics \cite{PMID_16918486}, and sequence-variation studies
\cite{PMID_17265721}.

\subsection{Microarray Technologies}

%@ Microarrays allow for the simultaneous measurement of gene expression...
%typical microarray experiment mRNA extracted, reverse trans... signals read...
%normalized etc.  @ Basic technologies differ but the general idea is the same,
%DNA is fixed in an array and hybrized with other DNA... spotted mixed samples
%vs.  single-channel systems...
There are two basic types of gene microarray technologies currently at the
disposal of researches: two-channel and single channel arrays.  Each has a
particular usefulness and differs in the implementation details but are
ultimately similar in approach.  In both systems, DNA is immobilized on a fixed
surface in a particular array pattern.  Each row and column intersection
contains a distinct, predetermined DNA sequence called a probe. The source of
the DNA can be cDNA clones or synthesized oligonucleotides.  Onto this fixed
array, a sample of either cDNA reverse transcribed from mRNA or, less
frequently, mRNA itself is hybridized to the array.  In either case the sample
is labeled with a fluorescent dye.  The sequence complementarity drives the
probes (fixed onto the array surface) and target cDNA or mRNA sample sequences
to anneal to each other.  The relative brightness of each probe is then
measured with a laser scanner and intensities for each are recorded.

In the case of the two-channel arrays, samples are always mixed with a
differently labeled reference sample while single-channel arrays only use the
actual sample.  The two-channel array expression values are typically
measured as log ratios of the two dye intensities while single-channel arrays use
 one intensity value. Regardless of the platform, the raw probe intensities
must be processed to yield interpretable results.  Arrays are often scaled, for
example, to all have an equal mean or median.  This normalization process makes
the comparison between arrays possible and there are many algorithms to choose
from.  Ultimately, the normalized data can be aggregated into an
two-dimensional expression matrix with samples versus probe expression values.
This output can be used in a variety of analysis methods.

%Figure \probefigure shows an overview of the process of sample preparation,
%hybridization, normalization, and data aggregation.

%@ possibility of using expression arrays as quantitative for diagnostics
%@ How they approach questions

\subsection{Research Approaches}

Microarray studies differ in the way they approach their core questions but
overall they can be divided into two general analytical techniques.  The first
is discovery driven where a fixed hypothesis is not necessarily created before
the experiment.  This method has been called \emph{pattern-detection} by
some and focuses mainly on the detection of interesting and previously unknown
relationships \cite{Dubitzky2003IMD}.  A common study may examine previously
undistinguished disease subclasses in a series of related samples.  A study by
Freije \ea\ used this approach to identify novel subclasses of malignant
gliomas \cite{PMID_15374961}. 

A second analytical approach to microarray experiments has been called
\emph{predictive modeling} and is concerned with learning a classification or
estimation function \cite{Dubitzky2003IMD}. In sample classification studies, also
known as supervised learning, the end goal is to predict the class labels of
samples. Generally, samples are divided into a training dataset and a
classification algorithm is trained to recognize the class labels. The
performance of the classifier is often optimized using a k-fold
cross-validation strategy.  This process is also commonly used to assess the
error rate in the classification model while the classification of the test set
is typically used to assess overall accuracy of the built classifier.

% FIXME: just use the Glolub example
Supervised learning approaches allow for new samples to be classified and 
can be a powerful tool for understanding sample differences. For example, a
researcher might have a collection of breast cancer and normal tissue samples.
A classification method could be applied to learn to predict the state of new
samples based on a subset of the expression data.  Subsequent new breast tissue
samples could then be classified into either a ``normal'' or ``cancer'' class using
this trained classifier. The complex question of how to choose genes for
classification (or other tasks) is the focus of the rest of this work.

%@ unsupervised methods let you do...

%@ supervised methods let you do... training set, test set classification
% FIXME: supervised learning is also known as induction (inductive inference or
% inductive learning)

%@ feature selection critical to these types of analysis... example of clustering
%@ To deal with the dimensions of microarray data researchers have focused on
%feature seleciton tasks. Deals with the problem of adding uninformative
%features, the search space increases exponentially.
One issue common across most analysis methods for microarray data is the
``curse of dimensionality'', where relatively few samples are available
compared to the number of measured features \cite{Bellman_1957}.
Features in microarray data refer to genes whose expression patterns are
measured across samples.  Typical microarray experiments include a few dozen
to, at most, a few hundred microarrays hybridization for very large studies.
However, each microarray measures thousands, typically tens of thousands, of
gene expression values.  Only a handful of the thousands of gene expression
patterns measure might be critical for the classification of samples.  Using
the expression information for all genes is undesirable since this increases
the computational complexity and run time for a classifier.  It could also
affect accuracy as previous work has shown that classification algorithm
performance decays as irrelevant features are added to the dataset
\cite{john94irrelevant}. 
% For many algorithms, the runtime increases dramatically with the number of
% features involved, sometime exponentially, making such algorithms intractable
% for microarray data.  which the most informative genes are selected that best
% differentiate between the sample states. This process is necessary for many
% sample classification algorithms due to the dimensionality problem with
% microarray datasets.  Typically, samples sizes range between a dozen to a
% hundred yet the measurements for each genes are in the tens of thousands.
% Many genes in a microarray experiment are uninformative and their use in a
% given classification proves pointless at best and, at worst, detrimental.
% Feature selection addresses this issue and is a critical step in classifying
% samples effectively.

\subsection{Feature Selection}

%@ statistical methods for feature selection
%@ Feature seleciton is a heuristic search problem w/ each state specifying a
%feature subset... different ways to proceed, greedy search etc... Two major
%approaches, explicit feature selection or Feature Weighting methods...
%Explicit feature selection can take two approaches, filter or wrapper
%methods...
Feature selection is a heuristic search problem where the goal is to eliminate
irrelevant features and retain relevant ones \cite{PMID_15219288}.  For microarray data,
this corresponds to selecting the best genes for the task at hand and removing
from consideration the vast majority of genes that are uninformative. One of
the most common applications of feature selection is the identification of
differentially expressed genes.  These are genes whose expression levels vary
significantly between two (or more) sample classes.  There are many different
ways to approach this problem and each of the algorithms have their own
strengths and weaknesses.  Almost without exception some feature
selection technique is used in microarray data analysis due to the sheer amount
of data, its general noisiness, and the technical limitations imposed by many
subsequent analysis algorithms.

\subsection{Explicit Feature Selection}

There are two general types of feature selection strategies
\cite{Xing_2003,PMID_15219288}.  The first, and most common, are the explicit feature
selection techniques.  These approaches focuses on finding a good subset of
features while removing the rest. The second general class are weighting
methods which continue to work with all features in a microarray dataset but
weight each one according to its relevance.  Often times feature weighting
is integrated into a classification algorithm.  As a general
feature selection technique, it is less commonly applied in microarray analysis
since it does not reduce the dimensions of the data.  Explicit selection
methods will be the focus here because of this limitation.

% FIXME: add refs from sections at end of paper

%@ Explicit search can include discretization, markov blanket filtering,
%decision tree, etc.
The explicit feature selection algorithms can be further divided into filtering
and wrapper methods. Filtering methods are much more common in microarray
analysis. The approach calculates a statistic for each gene's expression levels
and this score is used to sort the gene features by relevance.  Examples
include cutoffs for the variance of individual gene expression patterns, fold
change minimums between groups of samples, or calculated metrics such as the
\emph{P}-metric used by Golub \ea\ which measures the amount of difference
between the means of the groups normalized by the variation of the two groups
($\mu2 - \mu1 / \sigma2 - \sigma1$). The advantages of these filtering metrics
are their extreme simplicity and the speed with which they can be calculated.
Depending on the subsequent analysis, a different cutoff could be used to
produce a feature subset of any appropriate size.  Many alternative methods
have been examined in the literature and each have their strengths and
weaknesses.  An exhaustive overview of all feature selection algorithms applied
to microarray datasets is outside the scope of this work; however, several
methods are commonly used and described below. 

%FIXME: may want to say do they rank the genes or provide p-value for cutoff for each of these methods

% FIXME: should include references for application of each method to MA

%@ Common filtering algorithms and who made them: fold change, etc

%@ ___continuous___

%@ fold change

The simplest method for identifying differentially expressed genes is fold
change.  This is not a statistical test but, instead, uses expression value
thresholds to determine which class a sample belongs in.  The process works by
finding the means for a given gene's expression levels in two classes. The
genes whose mean difference is greater than a certain threshold (such as
2-fold) are selected \cite{derisi1997ema}.  This method is extremely simple and
suffers from bias when the microarray data are not normalized properly.
Efforts have been made to correct the problems but the fact remains, fold
change is not a statistical test and cannot report a confidence score or
probability \cite{yang2002wfa}.

%@ t-test and variations like t-tests

The most ubiquitous statistical measure to identify
differentially expressed genes, and hence perform feature selection, is the
\emph{t}-test.  Compared to fold change, this method allows for a probability
to be calculated for features which can be ranked accordingly allowing for more
accurate feature analysis \cite{PMID_164182572}. The Student's \emph{t}-test is
often employed in microarray data analysis rather than the traditional paired
\emph{t}-test.  This is due to the fact that many microarray studies sample
from different sources rather than the same source at different time points.
This makes the use of the paired \emph{t}-test inappropriate and the Student's
\emph{t}-test much more commonplace
\cite{ma2003gep,bueno2004dtp,PMID_14871811}. Regardless of the particular test
used, these methods allow for the calculation of \emph{p}-value which expresses
the chance a random sampling from the same normal distribution would produce
the observed difference in expression between the two classes. In order to
apply the \emph{t}-test the following must be true: (1) the two samples are
independently drawn from the source population, (2) the variance of the two
groups must be equal, and (3) the values for members from each must be normally
distributed.

\begin{equation} \label{ttest}
t = { \mu_1 - \mu_2 \over s_{\mu_1 - \mu_2}}
 \ \mathrm{where}\ s_{\mu_1 - \mu_2} = \sqrt{{\left({n}_1 - 1\right) s_1^2 + ({n}_2 - 1) s_2^2  \over {n}_1 + {n}_2 - 2}\left({1 \over n_1} + {1 \over n_2}\right)}
%t = {\overline{X}_1 - \overline{X}_2 \over s_{\overline{X}_1 - \overline{X}_2}}
% \ \mathrm{where}\ s_{\overline{X}_1 - \overline{X}_2} = \sqrt{{\left({n}_1 - 1\right) s_1^2 + ({n}_2 - 1) s_2^2  \over {n}_1 + {n}_2 - 2}\left({1 \over n_1} + {1 \over n_2}\right)}
\end{equation}

Equation \eqref{ttest} shows the test statistic calculation for the Student's
\emph{t}-test where $s^2$ is the unbiased estimator of the variance, ${n}_1$ is
the number of features in class 1 and ${n}_2$ is the number of features in
class 2.  Alternative methods exist that do not require the sample classes to
have identical variance such as the Welch \emph{t}-test \cite{PMID_142249476}.
%This allows for comparisons between sample classes with different variance.
There are many other modified versions of the \emph{t}-test and each have
particular conditions under which they are appropriate.

%@ signal to noise ratios

Miroslava \emph{et al.} describe another simple approach for distinguishing
between sample groups using signal to noise ratios (SNR)
\cite{PMID_164182572,PMID_12933571,yeang2001mcm,PMID_12384549}.  These methods
include a calculation for each feature that measures the differences in their
class means normalized by the variation of expression seen in each group.  This
criteria assures that features with highly different means between the sample
classes and with low variance within each sample class are identified.  Many
studies have used such an approach including Golub \emph{et al.}, a seminal
work that identified gene expression differences between leukemia subtypes
\cite{PMID_164182572,PMID_12933571,yeang2001mcm,PMID_12933571}.

\begin{equation}\label{pmetric}
P = { \mu_1 - \mu_2 \over {\sigma_1 - \sigma_2}}
\end{equation}

The calculation of the SNR \emph{P}-metric is shown in equation \eqref{pmetric}
where $\mu_1$ and $\mu_2$ are the means and $\sigma_1$ and $\sigma_2$ are the
standard deviations for a gene in class one and two.

%@ significance analysis of microarrays (SAM)
% maybe not this one since it uses FDR as an integral part of the process

Significance analysis of microarrays (SAM) identifies differentially expressed
genes between sample classes by performing a series of gene-specific test
similar to the \emph{t}-test.  SAM uses a \emph{d}-score which is the mean
difference between sample classes for a gene normalized by the standard
deviation of repeated measures for the gene \cite{tusher2001sam}.  A weighting
term is added to the pooled standard deviation used in SAM to prevent
overestimating the significance of genes with low variance.  The method
compares the \emph{d}-score calculated for each feature with the average
\emph{d}-score calculated with randomized sample labels.  The \emph{d}-scores
observed for each feature are compared to these expected \emph{d}-values and a
cutoff is determined by plotting these values. SAM is appealing because it
incorporates information of the SNR methods with a concept of controlling false
discovery rates (FDR) \cite{efron2001eba}.

% \begin{equation}\label{sam}
% 
% \end{equation}
% 
% Equation \eqref{sam} give the \emph{d}-value and weighted pooled standard
% deviation formulas.

%@ non parametric 

%@ Wilcoxon rank sum test (non-parametric)
% FIXME: the Mann-Whitney is the equivalent for t-test for two independent
% the Wilcoxon is the equivalent for the t-test for two correlated samples

The previous methods are parametric in nature and assume the data involved can
be modeled with a given distribution, most typically the normal distribution.
However, in the case of microarray data and many other types of real-world
data, the assumption of normality can be problematic.  Non-parametric
techniques can be used in these cases to arrive at ultimately the same sort of
information, namely a \emph{p}-value, without having to assume the data fits a
given distribution.  An example is the Mann-Whitney U-test which is a
non-parametric equivalent of the \emph{t}-test for two independent samples.
Rather than working with gene expression levels directly, the Mann-Whitney test
uses rank-transformed data expression.  This approach, along with other
non-parametric tests, have been used in several cancer classification studies
\cite{troyanskaya2002nmi,park2001nsa,dettling2002scg}. 

% FIXME: equation and algorithm explanation needed Equation \eqref{wilcox}

%@ multi-conditional

The feature selection methods examined thus far examine datasets with two
classes.  Often times, though, multiple classes are present and the end goal is
to identify genes differentially expressed between the categories.  The
one-against-others approach can be used to effectively turn a two-class feature
selection algorithm into a multi-class method.  In this approach the feature
selection algorithm is repeated for every class in the dataset and every gene
is assessed for its ability to distinguish the class labels of one class versus
all others.  The end results are features ranked based on their ability to
distinguish the class labels.  

%@ ANOVA (t-test for more than two classes)

The generalization of the \ttest\ to comparisons of more than two distributions
is the one-way analysis of variance (ANOVA).  This test allows the means of an
arbitrary number of groups to be compared assuming they each are normally
distributed.  ANOVA produces an \emph{F}-statistic that measures how much a
given gene's value varies within classes compared to between the sample
classes.  This can be converted to a \emph{p}-value and used to assess the
importance of each gene \cite{cui2003std,kerr2000avg,lee2002mmg,yang2003sma}.

%@ Correlation coefficient

Genes that distinguish sample classes have also been identified in several
studies through the use of a correlation coefficient.  This feature selection
method ranks genes based on how well their expression patterns correlate with
the sample classes.  Features with the greatest correlation to sample classes
are ranked highest.  Many correlation methods exist, the most commonly applied
to microarray data is the Pearson correlation coefficient. Improvements to the
method to deal with the problem of multiple testing have also been described
\cite{dudoit2002smi}.

% \begin{equation}\label{pearson}
% \Delta
% \end{equation}

% Equation \eqref{pearson} gives the general Pearson correlation coefficient.


%@ ___ discrete ___

%@ filtering by mixture modeling

The feature filtering methods discussed so far are typically used with
continuous gene expression data.  Other algorithms are tuned to work with
discretized data such as ``present'' or ``absent'' expression values. Gaussian
mixture modeling looks at the actual distribution of gene values and attempts
to model states in the data.  For example, a gene that is primarily expressed
in one of two states (for example ``cancer'' or ``normal'') can be modeled as
two overlapping normal distributions using the EM algorithm
\cite{dempster1977mli}.  A univariate mixture model with two components can be
used to model a gene that is expressed in two distinct levels.  A
mixture-overlap probability can be used to evaluate the goodness of fit for the
mixture model and, essentially, sort genes based on how well they discriminate
the two states.

% FIXME: add more

%@ Information gain and the other rank gene methods

Information gain is another technique for measuring how well a given feature's
expression pattern agrees with the partitioning of samples into distinct
classes. The first step in this process is to discretize the continuous
expression data into distinct states.  This can be accomplished using the
Gaussian mixture modeling described previously.  Information gain then measures
the degree of relevance of a feature to the sample classes.  The greater the
information gain, the more relevant the probe is to the partitioning of samples
\cite{cover1991eit}. This process can be described with the following equation:

\begin{equation}\label{sam}
I_g = H \left (  P \left ( S_1  \right ), ..., P \left ( S_C  \right )  \right ) -  \sum_{ k=1 }^{ K } P  \left ( E_k  \right ) H  \left ( P \left (  S_1 | E_k \right ),..., P \left (  S_C | E_k \right )  \right )  
\end{equation}

Where the sample classes are given by $S_1,...,S_C$, the feature being examined
has partitioned the samples into classes $E_1,...,E_K$, and the probability of
class $S_C$ given the predicted class of $E_K$ is given by $P \left ( S_C | E_k
\right) = P \left ( S_C  \cap  E_k \right) / P \left ( E_k \right) $.  $H$ is
the entropy function given by $H = \sum_{ i=1 }^{ c } - P_i log P_i$ for the
entropy distribution of $\left\{ P_1,...,P_c \right\}$.

One problem with information gain is that many similar features can all score with
an equally high value.  The result is a highly redundant
feature subset with an overrepresentation of certain gene families.  This can
ultimately compromise the effectiveness of subsequent classification algorithms
that use the output from an information gain feature selection filter.  The
technique of Markov blanket filtering circumvents this problem by eliminating
redundant features \cite{koller1996tof}. 

% FIXME: can also mention PCA?
% FIXME: add more of the survey papers!
%@ summarize other filter methods quickly to show diversity

The methods documented here are just a few of the many filtering feature
selection algorithms developed for general machine learning problems.  Many
more examples of adaptations of existing feature selection techniques or the
development of novel techniques for microarray data can be found.  A survey of
feature selection algorithms was profiled by Inza \ea\ where they examined the
performance of six total algorithms in a classification task, four of which
used discretized gene expression levels \cite{PMID_15219288}. These included
Shannon-entropy, Euclidean-distance, Kolmogorov-dependence, and the
Kullback-Leibler statistic.  For feature selection with continuous expression
data they used a modified version of the SNR \emph{P}-metric used by Golub \ea\
along with a \emph{t}-score based on the \ttest. Another survey by Chow \ea\
examined three feature selection algorithms; median vote relevance (MVR), naive
Bayes global relevance, and mean aggregate relevance (MAR) which is similar to
the \emph{P}-metric \cite{PMID_11242594}.  Jeffery \ea\ provided a
comprehensive overview of some of the most popular feature selection algorithms
used to date, many of which were described previously.  These included SAM,
between group analysis, ANOVA, template matching, empirical Bayes statistic,
MaxT, area under a ROC curve, rank products, the \emph{t}-statistic, and fold
change \cite{PMID_154652968}.  Liu \ea\ examined the performance of
Shannon-entropy, the Chi-Squared (${\chi}^{2}$) statistic, along with
modifications of the \emph{t}-statistic and \emph{P}-metric
\cite{PMID_14571374}.  In another survey, Lui examined the use of feature
selection in the context of drug discovery and included five methods:
information gain, mutual information, the ${\chi}^{2}$ statistic, odds ratio,
and a simplified ${\chi}^{2}$ statistic called GSS.  Finally, Su \ea\
implemented eight algorithms for feature selection in their RankGene program
\cite{PMID_146536352}.   The methods included the Twoing rule, information
gain, gini index, max minority, sum minority, and sum of variances methods.
Despite their diversity, these filter feature selection algorithms share the 
common end goal of ranking gene features in order to simplify microarray data and to 
identify the most important features.
% These have well defined characteristics and are commonly applied to feature
% selection problems.  

% For each feature examined they attempt to divide the range of expression
% across the samples into two disjoint sets, one with as a "high" expression
% set and the other as "low".  The samples were then annotated into one of the
% two samples classes based on which of these sets they fall into.  These
% algorithms differ in the error scoring function they use, for example the sum
% minority rule counts the total number of errors in each class.  The net
% effect however is the same, the probe sets are sorted based on their ability
% to correctly segment samples into their respective classes.

% Could mention other common packages here too!

%@ Wrapper Method is limited because it is an NP-hard.  hill-climging,
%best-search first, etc...  ORDERED FS is a combo wrapper and filter method.
Wrapper feature selection methods are quite different from the explicit feature
filtering techniques outlined previously.  They rely on indirectly
assessing feature subset selections through cross-validation with a particular
classification algorithm \cite{Berrar_2003}.  These methods are much more
computationally expensive than simple gene ranking algorithms due to the search
space of possible feature subsets being very large. Enumeration over the
complete $2^n$ possible feature subsets, where \emph{n} is the number of
features, is NP-hard making an exhaustive search impractical for even small
values for \emph{n}. The focus of these methods, then, is how to efficiently search
the possible subgroups of features while at the same time finding a near
optimal subgroup.  Hybrid methods exist, such as the ORDERED-FS algorithm, that
first rank features using an explicit filtering algorithm and subsequently
explore the possible feature space to find a well-performing feature subset
through cross-validation \cite{Ng_1998}. These benefit from improved
performance but are still, like all wrapper methods, much more computationally
challenging than feature filtering statistics such as the \ttest\ or information
gain. Because of their infrequent use in the domain of microarray research,
wrapper methods will only be covered breifly in this overview.

\subsection{Feature Selection Performance}

%@ no one specific method dominates as being best
%Notes
%@ Comparing feature selection algorithms is difficult... Many use
%classification performance as a metric.  
A common approach to compare feature selection algorithms is to test their
ability to identify meaningful and important features and an ideal task to
evaluate this is sample classification.  Many of the comparisons of feature
selection algorithms mentioned previously use classification accuracy, or a
related concept such as cross-validation error rates, as a ranking mechanism
for feature selection algorithm performance.  The classification problem, more
generally referred to as supervised learning, is typically divided into two
phases: (1) training the classifier and (2) using the classifier to classify
a test set.  Feature selection begins the process by defining which gene
features are to be used in both the training and testing phases.

% Go on to mention: 1) no clear performance winner, 2) depends on the dataset
% and sample classes examined, 3) wrapper methods generally perform better but
% are less often used.
Given the many comparisons between feature selection algorithms, both wrapper
and filter methods, no clear leader has emerged as being superior.  Instead,
feature selection algorithms vary in their ability to identify features that
perform well in classification tasks.  This variation in performance seems
attributed to the particular dataset and class labels being examined as well as
the induction algorithm being used \cite{guyon2002gsc,PMID_15087314}.  The
wrapper methods generally perform better than simple feature filtering methods
\cite{Ng_1998,Berrar_2003}.  This can be attributed to the nature of the
algorithms since they each specifically identify features that perform well in
the classification algorithm used for testing.  This leads to the selection of
features that are particularly useful in the induction algorithm being used.
However, this performance premium comes at a cost.  Wrapper methods are much
more computationally demanding than the simpler filtering methods.  As a
result, most microarray analysis is performed using the faster filtering
methods.  

%@ The current commonly used feature selection algorithms, most are simple
%filtering technqiues, are typically based on individual genes being evaluated
%one at a time.  Nothing wrong with this but there are biological reasons why
%examining multiple genes would be beneficial.
The vast majority of feature selection techniques, regardless of whether they
use a wrapper or filter approach, consider the contributions on only one gene
at a time. There is no doubt that, despite this limitation, many interesting
and important individual genes have been identified that are directly related
to the underlying biology of the samples being examined.  However, the complex
biological systems being studied are not just limited to the effects of
independent, single genes.  Instead, even in the simplest biological systems
the interactions between many gene products ultimately determine the phenotypes
and classes observed in the samples.  Ignoring the interdependency of gene
expression patterns has resulted in missed opportunities to understand more
complex gene relationships.

%@ PCA and Bo et al. technique show that feature selection with multiple gene
%info can be beneficial to the subsequent classifier.  So look at two, or more,
%genes at a time can boost classification performance.  Yet there are some
%issues, PCA can't reextract the genes contributing to the component.
Some studies have looked at using multiple gene expression information in
feature selection.  B{\o} \ea\, for example, examined pairs of genes in a
feature selection algorithm that made use of the diagonal linear discriminant
(DLD) method to evaluate how well pairs together separated sample classes
\cite{PMID_11983058}.  Others have shown the first several principle components
from the principle component analysis (PCA) of microarray data can be
effectively used to identify components that combine gene information to
distinguish sample classes \cite{PMID_173481048,PMID_11126130,PMID_10963673}.
Yet in both these cases the actual relationships between pairs, or large sets,
of genes is hard to interpret from a biological perspective. In the case of
PCA, it is difficult to recover meaningful gene relationships from the
principle components.  Still, these methods show the utility of examining the
information from two (or more) genes at a time.

\subsection{Logic Analysis}

%@ Logic analysis is another way to view gene relationships whose
%interpretation is more clear.  It was originally used as a phylogenetic
%approach to understand... Breifly, this worked like...
%@ A method to adapt logic analysis to microarray data was explored by Bowers
%et al.  The process worked by...
Logic analysis was a proposal by Bowers \ea\ as an extension of
the phylogenetic profiles approach to understanding genes' presence or absence
across different prokaryotic genomes \cite{PMID_144923264,pellegrini1999apf}.
In phylogenetic profiles analysis, gene presence and absence vectors are compared and
those with the most similar vectors are predicted to interact in the same
pathway.  Similarly, logic analysis looked for relationships between gene
presence or absence patterns across prokaryotic organisms. However, the vectors of two
genes were logically combined and then compared to a third binary gene vector.
For example, one possible logic relating two gene profiles to a third would be
if a particular gene is present in genomes A and B, then it is also present in a
third genome C.  Other kinds of logical relationships --- involving negation
and ``or'' instead of ``and'' --- lead to eight distinct types of logic relating
two binary profiles to a third.

This technique allowed for the identification of novel gene relationships that
were previously missed by the standard phylogenetic profiles approach
\cite{PMID_144923264}.  One example from the Bowers \ea\ dataset examined the
synthesis of aromatic amino acids through the shikimate pathway which includes
a branch point.  Either one of two shikimate kinase proteins (A or B) are
required for the synthesis of the end product by excitatory postsynaptic
potential synthase (EPSP). The logic relationship inferred in this example can
be expressed as C (EPSP) is present if and only if one of either kinase A OR
kinase B is also present.  Figure \ref{phylo_la} shows another example of
applying logic analysis to prokaryotic genomes in order to identify logic
relationships.

\subsection{Feature Selection with Logic Analysis}

This approach has been adapted to microarray analysis and looks at the logic
combination of gene expression states (``off'' and ``on'') in relation to the sample
class annotations.  In this way, pairs of genes are ranked with respect to
their correspondence to sample class.  The output from logic analysis, like any
feature selection filtering algorithm, can be used with a variety of
classification algorithms, for hypothesis generation, for the identification of
differentially expressed gene pairs, and other methods.

Samples are first annotated by creating a binary matrix in which a ``0''
indicated the sample belongs to the first category while a ``1'' indicates it
belongs to the second.  Datasets with samples in more than two classes are
divided into \emph{n} trials using a one-against-others approach where \emph{n}
is the number of classes.  Gene expression levels in each sample are converted
to binary ``presence'' or ``absence'' calls.  For each class annotation vector,
all possible logical combinations of gene binary expression states are
compared.  This identifies pairs of gene expression states that, when logically
combined, corresponded with the sample binary class vector well. For example,
in Figure \ref{gene_la}, the logical combination of the binary gene expression
states for glial maturation factor $\gamma$ (GMFG) and glucose transporter 10
(GLUT10) are closely related to a poor prognosis state in tissue samples from
malignant gliomas (HC\_2B). This can be expressed with the logic relationship:
the sample class is HC\_2B if and only if GMFG and GLUT10 are expressed
\cite{PMID_170746732}.

In many ways this is similar to logic regression in that binary states of
features are being logically combined and compared to binary sample class
vectors \cite{ruczinski2004eih}.  However in logic regression, combinations of more
than two features are compared.  A stochastic, simulated annealing approach is
used to sample from the enormous number of possible feature combinations.  This
approach is based on converting logic relationships to tree structures and
making specific additions and deletions to the classification tree.
Ultimately, the end result is a regression model that relates the binary
expression pattern of several features to the sample state.  While logic
regression can explore much more complex feature relationships than logic
analysis, it comes at the expense of greatly increased computation time and an
inability to exhaustively explore the search space.

\subsection{Logic Analysis Scoring}

%@ The thought was to use LA as a generic feature selection algorithm to select
%gene pairs that are informative together when previously they were
%uniformative.  
The logic analysis feature selection algorithm uses two common scoring
techniques to evaluate the significance of each pairwise gene expression combination
matching to the class vector.  First, a hypergeometric \emph{p}-value is 
calculated to evaluate the likelihood that the combined gene expression vector randomly
matches the class vector as well as observed.  The hypergeometric
\emph{p}-value is used to eliminate combinations of genes that are likely to
match the class vector just by chance. The second scoring technique is based on
an uncertainty (U) score which ranges between 0 and 1. An uncertainty score of
0 indicates that one vector does not explain another while a score of 1
indicates that one vector explains another perfectly.  Based on previous work, a
threshold of 0.3 and 0.6 are typically used for the uncertainty score of the
individual gene expression vectors and the logically combined gene expression
vector respectively \cite{PMID_170746732}.  These cutoffs ensure that genes that
individually explain sample classes well are removed and those that only
explain the classes well when logically combined are identified.

\begin{equation}\label{laeq1}
U \left[ c,  f \left( a, b \right)  \right]  \gg U \left( c | a \right) \ \mathrm{ and } \ U \left( c | b \right)
\end{equation}

\begin{equation}\label{laeq2}
\ \mathrm{where } \ U \left( c | a \right)  =  \frac{ H  \left( c \right) +  H  \left( a \right) -  H \left( c , a \right)  }{ H  \left( c \right) }
\end{equation}

\begin{equation}\label{laeq3}
\ \mathrm{and } \ H \left (  a \right ) =   \Sigma p \left (  a \right ) \ln  \left (  p \left (  a \right )   \right ) 
\end{equation}

\begin{equation}\label{laeq4}
\ \mathrm{and } \ H \left (  c,a \right ) =   \Sigma \Sigma p \left (  c, a \right ) \ln  \left (  p \left (  c,a \right )   \right ) 
\end{equation}

Equations \eqref{laeq1} - \eqref{laeq4} illustrate the uncertainty score criteria and calculation.

%@ Findings from this use of Logic Analysis to analyze microarray data show
%that the relationships are non-random and interestingly relate genes that
%previously weren't related...
% In this
% process, gene expression vectors were first converted from continuous
% expression data to binary transcript presence or absence calls.  All possible
% pairwise combinations of two genes were examined and for each pair all eight
% possible logical operations were computed. These logically combined gene
% vectors were then compared to the sample class annotation binary vectors.  In
% this way, pairs of genes that matched the sample class annotations well when
% logically combined were identified. Care was taken to eliminate genes that
% individually explained sample classes well.  This ensured novel gene-gene
% relationships were identified that were distinct from other analysis methods.

%@ The output are pairs of genes specifically related to sample state.
We have previously shown that logic analysis can identify non-random gene and
class relationships form a microarray dataset \cite{PMID_170746732}.  In
Chapter \featureselchapter, results from adapting logic analysis to feature
selection are explored and compared to other feature selection techniques.  Our
results are interesting because they show that genes previously missed by other
feature selection techniques can, when logically combined, yield similar sample
classification performance results. This is even more surprising considering
logic analysis not only selected genes that are individually poor sample
classifiers but its also used binary data compared to continuous data for the
other feature selection techniques examined.  Furthermore, an appealing quality
of feature selection with logic analysis is that genes identified are related
with specific logic types leading to the creation of hypotheses regarding gene
expression pattern. This provides more information than other feature selection
techniques including those that examined gene pairs or composites.    

\subsection{Further Work}

Chapter \logicanalysischapter\ provides additional details about the application
of logic analysis to microarray data.  This includes work supporting the claim
that logic analysis is able to identify non-random gene expression
relationships.  Chapter \featureselchapter\ explores the use of logic analysis
as a general mechanism to discover meaningful gene relationships in additional
microarray datasets.  It compares the performance of logic analysis to other
feature selection algorithms and assess the ability of these features selected by
logic analysis to classify samples in a supervised learning test with a variety
of classification algorithms.





\newpage

\section{Informatics}

%@ The role of informatics in biology is dramatically changing the discipline
%and opening up the applications of high-throughput, multiplexing technology
The field of biology is in the process of rapidly changing from a discipline
that largely focuses on bottom up approaches to one that produces and analyzes
vast amounts of high-throughput, high-dimensional data. New emerging
technologies such as DNA microarrays, mass spectrometry, and rapid genome
sequencing mean that a biologist can now study entire organisms rather than just
one or two genes at a time.  This transition has created exciting opportunities
for integrating a wide variety of previously separate research areas including
engineering, statistics, computer science, and other disparate subjects.  The
end result is a field that is swiftly adopting new technologies and approaches
to analyze these highly integrated and challenging datasets. 

%@ Biologists today are facing a huge problem of data sharing, organizing,
%analysing, etc.  Needs solutions to a host of problems.
% Biologists today face huge problems of storing data and organizing large
% datasets such as microarrays.  The typical microarray hybridization experiment
% on the Affymetrix U133A platform contains approximately 22,000 features and
% individual raw image files can take up to 150Mb each of storage space each.
% Pre-processed data files that summarize image files take up less space but
% large datasets can still become difficult to work with.  The Celsius project,
% for example, mirrors public microarray respositories allowing for research into
% comparisons between arrays from thousands of experiments.  The processed
% expression file for this conglomerate dataset is over 5Tb which is
% prohibatively large for most research labs to duplicate. Furthermore, the
% actual analysis of very large datasets from mass spec or microarray
% technologies can tax computer RAM and CPU capabilities.  The all-by-all
% comparison of probeset correlation, for example, is currently intractable on
% the Celsius dataset on current hardware without first dividing the task into
% smaller steps. For problems where a natural parellel solution is not possible
% limitations of CPU and RAM during analysis become acute issues.

\subsection{Informatic Challenges in Biology}

%@ These core problems break down into several related but distinct activites
% The informatics problems facing biologists can be divided into several distinct
% classes. Issues of working with microarray data can be thought of
% as a microacosum of the common types of problems biologist are experiencing as
% a result of new technologies.  A
% scientist studying gene expression patterns with a large number of microarrays
% experiments must be able to store both the raw and processed results.  These
% might include image files, processed expression data, normalized gene
% expression values, and other derived data.  Experimental annotations must be
% collected that clearly identify samples, how they were treated, and the
% parameters of the physical assay. Next, techniques for analyzing the data must
% be standardized.  This could include common algorithms or best practices for
% working with the data.  In microarrays, many competing analysis techniques are
% available and, regardless of the choice, a common way of accessing them is a key
% concern.  Furthermore, results can change with subsequent releases of new
% software or datasets. Since many scientific questions can take months or years
% to answer, keeping track of what resources were used to produce a given result
% are critical.  Microarray data can be normalized in a number of
% different ways with a variety of algorithms designed with similar but slightly
% different assumptions. Finally, sharing information is the final activity of
% most academic research.  Journals have become more sophisiticated in moving
% articles to completely digital distrubution and archiving models.  However as
% datasets continue to grow access via programatic interfaces is more and more
% important.  This allows for verification of existing results and also the
% synthesis of new questions and research directions.  Microarray data deposition
% in public repositories is a requirement for most journals yet automatic
% mechanisms for working with this data are lacking and requirements for what
% constitues primary data in this field are vague at best. The challenges facin

%@ Commonly applied solutions across the fields of information technology and
%computer science (collectively refered to as informatics) can meet many of
%these needs.  A hybrid approach is needed that blends the best resources from
%a variety of data driven desciplines with the highly parellel nature of modern
%genomics.
The issues facing biologists include storage, annotations, analysis, sharing, and
standardization.  All have roots in the highly technical and interdisciplinary
nature of modern biology.  Bioinformatics, computational biology, and other
related ``new'' fields take an interdisciplinary approach to solving these
challenges.  Commonly applied solutions across the fields of information
technology (IT), computer science, and engineering (collectively referred to
here as ``informatics'') can meet many of these needs.  A hybrid
approach that blends the best resources and practices from a variety
of data driven disciplines with the highly parallel nature of modern biology is needed.
Of particular note, the open source movement has achieved a high level of
success in creating free and flexible software that is philosophically aligned
with the principles of open, transparent, and freely accessible research.  It
is no wonder that most bioinformatics projects use open source software, such
as the Linux operating system, for the bulk of their work.  Solutions for many
of the technical challenges surrounding the transforming field of biology can
be found in the open source movement.

In the following sections, possible solutions to some of the specific informatics
challenges facing biology today are described. Summaries of the open source
bioinformatics projects in this dissertation are available in Table \ref{projects}.

%@ Informatic solutions borrowed from computer science include...
%@ These solutions can be adapted to the core problems...

\subsection{Integrative Solutions}

\subsubsection{Storage}

%@ storage -> datatbases
Storage solutions borrowed from computer science and information technology
include physical medium on which to store the data, such as hard drives, and also a
database system to structure the data in accessible and searchable contexts.
Hard drive storage systems have advanced considerably over the decades in
response to the demand for safe, reliable, and cost effective ways of storing
large amounts of data.  Systems that link together many individual hard drives
into contiguous virtual volumes are available, making it possible to group
together large datasets in a common repository.  Technical details such as RAID
redundancy make the storage of critical scientific data secure and retrieval
speedy.  Database systems represent another strategy for meeting the storage
needs of bioinformatics projects.  Unlike hard disk-based solutions, databases
actively index information to improve the retrieval of structured data.
Advances in open source projects such as mySQL (\url{http://www.mysql.com}) and
PostgreSQL (\url{http://www.postgresql.org}), have allowed researchers to use
very high performance relational database systems in their research.

Many database solutions exist for representing biological data.  The generic
model organisms database project (GMOD, \url{http://www.gmod.org}) provides the
modular Chado schema for storing a wide variety of biological data.  This
schema has been used by a variety of projects including the Celsius project for
microarray datasets.  The Appendix describes the Celsius project,
the overall infrastructure behind it, and its potential widespread uses in
microarray data analysis.

\subsubsection{Annotations}

%@ annotation -> ontologies
Ontologies represent another key crossover from computer science, information
theory, and artificial intelligence research.  An ontology is a system of
related terms that is intended to encapsulate the classification of a system.
They are a way of describing and understanding a facet of the world and, as such,
serve as natural points of standardization.  As biologists have always faced
the problems of nomenclature and classification, ontologies are a natural
extension of these activities.  Many high-profile ontologies have been created
to annotate and classify a wide variety of biological concepts.  For
microarrays, much work has gone into defining standards (such as MAGEML and
MIAME) that structure the task of recording information about experimental
design \cite{brazma2001mim}. Extensive work has also resulted in the use of
ontologies to describe biological systems and the experiments performed on
them.  The MAGE ontology is of particular importance in the microarray field
but others exist such as the extremely useful gene ontology (GO) that aims to
structure the annotations on gene function and nomenclature
\cite{stoeckert2003mof,ashburner2000got}.

Many important projects have spearheaded the effort to establish ontologies for
the biological research community.  The open biomedical ontologies (OBO)
project provides a range of ontologies designed for use in the biomedical
fields.  Other related projects include GO, the MGED ontology, and the sequence
ontology (SO) \cite{stoeckert2003mof,ashburner2000got,eilbeck2006sot}.  Many
organization exist to support the development of new and existing ontologies
including the National Center for Biomedical Ontology
\url{http://www.bioontology.org}, MGED \url{http://www.mged.org}, and OBO
\url{http://obo.sourceforge.net}.  Ontologies from these projects have been
made available in Chado and are used extensively in the Celsius project to
annotate microarray experiments with disease type, tissue type, and other
important information.  The Celsius system is described in more detail in
the Appendix.  

\subsubsection{Analysis}

%@ analysis -> common APIs
In microarrays, many different algorithms exist that allow a researcher to
process the raw information from a hybridization into meaningful gene
expression data.  This process, called normalization and standardization, can
range from a simple scaling to sophisticated algorithms that model probe
hybridization efficiency and integrate data from multiple experiments to arrive
at a more accurate value. With many different options, the need for
standardizing a mechanism to interact with these algorithms becomes apparent.
Without common application programming interfaces (APIs), each distinct
algorithm would require its own input and output format and require many more
hours to be spent on the trivial task of data formating.  Instead, good
programming practices from computer science dictate coding abstraction and
encapsulation.  These ideas can be seen in the Bioconductor or Bioperl APIs for
biological data and algorithms \cite{stajich2002btp,gentleman2006bos}.  For
microarrays this translates to a common framework for accessing many different
normalization techniques with the same programmatic calls, only the flag
specifying what algorithm to be used changes.  Similarly, the Seq::IO section
of the Bioperl API allows sequence data to be read and written to and from a
variety of sequence file formats \cite{stajich2002btp}.  This allows programs
to access many different formats of data without needing to be rewritten.
Common APIs clearly are favored by both the open source community and the
bioinformatics community for these reasons.

\subsubsection{Sharing}

%@ sharing -> web services
Related to the concept of open APIs for interacting with biological data is the
concept of web services.  Sharing of biological data, whether it is raw,
primary data, or processed results, is of the utmost concern for biologist.
Yet journal articles provide a poor repository for large datasets.  Simply
downloading large data files is a better approach, yet many times only a few
pieces of information are needed and the downloading of an entire dataset is
unnecessary.  For example, if a researcher wants a particular protein structure
from the protein data bank (PDB, \url{http://www.pdb.org}) downloading the
thousands of structures in bulk from the PDB is unnecessary.

The emerging web services approach used across the Internet points to a better
solution.  This model includes technology such as the simple object access
protocol and REST concepts for interacting with an API remotely over the
hypertext transport protocol (HTTP) on which the web is based.  This idea is
that simple requests for data or information calculated on the fly can be made
by a researcher and the result is calculated or retrieved remotely and returned
over the Internet.  This type of approach is extremely flexible and can be used
in a variety of contexts to present biological data to other researchers. 

The distributed annotation system, or DAS, is a popular bioinformatics web
services project geared towards the sharing of genome annotations with the
larger research community \cite{dowell2004das}.  Organizations looking to share
genome assemblies, gene annotations, and other genomic features use DAS to make
this information available over the web.  Implementations of DAS use the
standard HTTP protocol and XML as an exchange standard.  The next version,
DAS/2, expands on the genomics focus of DAS by including capabilities to
exchange ontologies, download experimental assay results such as microarray
data, and perform on-demand sequence analysis such as BLAST.  The success of
DAS as a project is due to the ease of which scientists can utilize information
published with DAS.  Many clients exist, such as GBrowse and IGB, and the web
services model affords programmatic access to the servers \cite{stein2002ggb}.
This allows additional applications to be built on top of these public
repositories.  For example, the Celsius project web interfaces were created
on top of a DAS/2 server which provided the raw data.  These web tools let an
end user query the microarray data available via ontology annotations and
download the corresponding data in a variety of different processed forms. 

More information on web services can be found in Chapters \gdapchapter,
\gmodwebchapter, and the Appendix. Chapter \gdapchapter\ details
the creation of a web services designed to thread protein sequences onto know
structures with the end goal of identifying disulfide bonds.  Chapter
\gmodwebchapter\ examines a model organism website generation framework that
includes web services tools. 

\subsubsection{Standardization}

%@ standardization -> software packaging
A closely related concept to web services is the concept of software
standardization.  In the model of web services, a researcher can focus on the
analysis of data and its biological meaning rather than figuring out how to
store data locally.  This approach affords abstraction between the researcher
and the entity providing the web service, making it easier for others to
either validate existing work by performing the same analysis or expand on the
work using the same web services.  It allows researchers to easily standardize
a given dataset or analysis server.  Another technique familiar to all
computer users on standardization is the versioning of computer programs.  When
research is being performed on a particular dataset or with a particular
software program, it is extremely important to track which version was
used.  Otherwise it becomes impossible to replicate the work.  The idea of
software packaging, borrowed from the field of information technology, is of
key importance to bioinformatics.  In addition to simply versioning software,
many comprehensive systems exist for specifically tracking, installing, and
updating both software and data in a particular computing environment.  The
Linux system, for example, uses one of several different package managers to
perform this task.  

The Biopackages project looks to standardize many tools used commonly in
bioinformatics projects. It encompasses an automated build system that creates
software packages for particular Linux distributions.  These include packages
for APIs such as BioPerl and BioConductor, web services such as DAS/2, and
databases such as Chado.  Details of the construction, public availability, and
benefits of this standardization tool can be found in Chapter \biopackageschapter.

%@ I have worked on several software solutions for these (maybe just integrate
%these above) This can just be a "Chapter X shows..." paragraph that maps the
%topics above to chapters. I'm just going to include these above.

\subsection{Further Work}

%@ Future biologies will be driven by basic needs as biology continues to
%become integrated with other disciplines.
Biology will continue to be driven by the basic needs to store, annotate,
analyze, share, and standardize biological data and practices. As biology
continues to become more integrated with other disciplines and influences the
development of other fields at the same time, new and effective technologies
will continue to be developed. In this section, many facets of computer science
and information technology were explored as they relate to the common problems
facing biologists today and in the future.  The use of computer science and
information technology continues to expand in biological research and provies
new avenues to address challenges and demands of this transforming field.


% table of projects

\begin{table}[p]
\caption[Open source bioinformatics projects]{Open source bioinformatics projects that are presented in this dissertation.}
\label{projects}
\begin{tabular}{|l|p{3cm}|p{7cm}|}
\hline
\textbf{Project} & \textbf{Challenge} & \textbf{Description} \\
\hline
\hline
Celsius & storage, annotation, and sharing  & The Celsius project is a centralized repository for normalized microarray data. \url{http://sourceforge.net/projects/celsius} \\
Biopackages & standardization & A repository of pre-packaged bioinformatics software. \url{http://biopackages.net} \\
GMODWeb & sharing & A model organism database website generation tool. \url{http://turnkey.sf.net} \\
GDAP & analysis & A protein threading and disulfide prediction tool. \url{http://www.doe-mbi.ucla.edu/Services/GDAP}\\
\hline
\end{tabular}
\end{table}

\clearpage

% Figures

%\begin{figure}[p] \includegraphics[width=14cm]{figures/Sterner_Fig_1.pdf}
\begin{figure}[p] \includegraphics[width=14cm]{figures/g_fxn_of_t.pdf}

\caption[A theoretical graph of $\Delta$G as a function of temperature T]{A
theoretical graph of $\Delta$G as a function of temperature T.  The solid line
shows the curve for a mesophilic protein with a temperature of melting of
${T}_{m}$(m).  Thermophiles can increase their ${T}_{m}$ through a variety of
mechanisms that result in distinct shifts of the $\Delta$G curve.  Increasing
maximal $\Delta$G results in shifting the curve up and, as a result, increasing
${T}_{m}$ to ${T}_{m}$(t) (dashed line).  Shifting the curve to the right
(dash-dot line) or broadening the curve (fine-dash line) also increase the
${T}_{m}$.  Arrows indicate the maximal $\Delta$G values.
%Copyright 2001, reproduced from Sterner and Liebl \cite{PMID_11256505} with
%permission from Taylor \& Francis Inc.
}

\label{Sterner_Fig_1} 

\end{figure}

\begin{figure}[p] \includegraphics[width=14cm]{figures/Sterner_Table_1.pdf}

\caption[Comparisons of amino acid usage between the thermophile
\emph{Methanococcus janaschii} with several mesophilic \emph{Methanococcus}
species]{A comparisons of amino acid usage between the thermophile
\emph{Methanococcus janaschii} with several mesophilic \emph{Methanococcus}
species. Copyright 1999, reproduced from Haney \emph{et al.}
\cite{PMID_10097079} with permission from PNAS.}

\label{Sterner_Table_1} 

\end{figure}

% \begin{figure}[p] \label{Sterner_Table_2} \includegraphics[width=14cm]{figures/Sterner_Table_2.pdf}
% 
% \caption{A systematic comparison of mesophiles and thermophiles reveals trends
% in stabilization techniques.  The number of arrows (1, 2, or 3) indicates the
% significance of the effect (insignificant, moderately significant, or highly
% significant respectively). Copyright 2000, reproduced from Szilagyi and
% Zavodszky \cite{szilagyi2000sdb} by permission of Elsevier Science.}
% 
% \end{figure}

\begin{figure}[p] \includegraphics[width=14cm]{figures/Boutz_Fig_3.pdf}

\caption[The disulfide bond regulation system in \ecoli]{The disulfide bond
regulation system in \ecoli.  Enzymes involved in disulfide reduction are
colored blue, enzymes involved in disulfide formation are colored red, enzymes
involved in disulfide isomerization are colored yellow.  Arrows designate the
direction of electron flow.  Copyright 2006, reproduced from \emph{\dbthesis}\
with permission from D. Boutz \cite{boutz_thesis}. }

\label{Boutz_Fig_3}

\end{figure}

\begin{figure}[p] \includegraphics[width=6cm]{figures/Boutz_Fig_7.pdf}

\caption[The crystal structure of \emph{Pyrobaculum aerophilum}
adenylosuccinate lyase monomer reveals the presence of three intramolecular
disulfide bonds]{The crystal structure of \emph{Pyrobaculum aerophilum}
adenylosuccinate lyase monomer reveals the presence of three intramolecular
disulfide bonds.  Disulfide-bonded cysteines are illustrated as orange spheres.
Copyright 2006, reproduced from \emph{\dbthesis}\ with permission from D. Boutz
\cite{boutz_thesis}.}

\label{Boutz_Fig_7} 

\end{figure}


\clearpage

%\begin{figure}[p] \includegraphics[width=14cm]{figures/probefigure.pdf}
% \begin{figure}[p] \includegraphics[width=14cm]{figures/TadG.png}
% 
% \caption{}
% 
% \label{probefigure}
% 
% \end{figure}


\begin{figure}[p] \includegraphics[width=14cm]{figures/genomic_la.pdf}

\caption[Logic analysis of genes in prokaryotic genomes.]{ Logic analysis was
performed looking at the presence or absence of gene families across many
different prokaryotic genomes.  In this case the pilus protein TadG is present
if and only if TadC and TonB are both present in the genome.}

\label{phylo_la}
\end{figure}


\begin{figure}[p] \includegraphics[width=14cm]{figures/expression_la.pdf}

\caption[Logic analysis of binary gene expression data.]{ Logic analysis was
performed on binary expression data where pairs of gene expression states were
compared to the sample class vector.  Here the sample was a poor prognosis
cancer type if and only if the glial maturation factor $\gamma$ and glucose
transporter 10 were both expressed.}

\label{gene_la}

\end{figure}



\bibliography {thesis,la,dboutz}    % bibliography references
%\bibliographystyle {uclathes}
\bibliographystyle {plain}

