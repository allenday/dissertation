There are two basic types of gene microarray technologies currently at the
disposal of researchers: single-channel and multi-channel microarrays.  The
methods used and form of measurements made for these two types differ, but the
research questions that are addressable using either type are the same.  In
both systems, DNA is immobilized on a fixed surface in a specific and
reproducible pattern.  This is the \emph{microarray}, sometimes abbreviated as
an \emph{array}.  The pattern that allows distinct regions on the
two-dimensional space of the surface to be mapped to the specific DNA sequences
immobilized at that address is known as a \emph{array design}.  Each distinct
region on the array is known as a \emph{feature} and the DNA sequence present
at a feature is called a \emph{probe}.  The source of the probe that is
immobilized at the feature can be a cDNA clone deposited onto the microarray,
or oligonucleotides that are synthesized on the microarray \emph{in-situ} using
photochemistry and light masks \cite{XXX}.

Once a microarray has been fabricated according to the array design, a solution
of DNA fragments, or \emph{targets}, can be labeled with a fluorescent dye and
\emph{hybridized}, or allowed to chemically anneal to the microarray's probes.
The property of reverse complementarity between subsets of the solvent and
immobilized DNA drives the solvent DNA targets to localize near to their
corresponding immobilized probes.

After hybridization, the amount of target DNA hybridized to each feature on the
array is assayed.  A laser is used to excite the fluorescent dye associated
with the targets, and a digital camera is used to measure the amount of
fluorescence present at each feature, and at the specific wavelength of
fluorescence known as a \emph{channel}.  The amount of fluorescent light is
assumed to be a monotonic and increasing function of the amount of target DNA
hybridized to that feature.  In the case of single-channel assays, only a
single fluorescent dye is used.  In a multi-channel assay, multiple target
solutions are hybridized to the array, each labeled with a different
fluorescent dye.  Each digital image of a microarray's fluorescence pattern for
a single channel is known as an \emph{image}, and the process of observing and
recording this image is called \emph{acquisition}.

For a single-channel assay, the acquired image is processed to produce a matrix
of numbers that has the same dimensions as the array design, and whose values
correspond to the absolute intensity of light present at each feature.  A
similar matrix is produced for a multi-channel assay as well, but because of
the competitive hybridization between the multiple labeled targets, the values
reported are not absolute, but rather the intensity of each channel relative to
all other target channels at a given feature.  Typically one of the channels
corresponds to a \emph{reference sample} that is a common denominator to
several arrays that will be analyzed as a set.  This allows the relative values
for the non-reference samples on each array to be compared.

For both single-channel and multi-channel arrays, the next step is
\emph{quantification}, a series of processes designed to convert the ``raw''
fluorescence intensities acquired from the scanner into values that are usable
in higher-level data analyses such as classification.  Methods for
quantification of microarray data are an area of intense interest
\cite{XXX,XXX,XXX} because they can give very different estimates of the
quantity of DNA present for a given feature/channel, and thereby have a large
effect on the biological conclusions that can be drawn from analyses of those
estimates.  Quantification methods are discussed in greater detail in Section
\ref{Quantification}.

