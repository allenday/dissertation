%%% XXX
There are two basic types of gene microarray technologies currently at the
disposal of researchers: single-channel and multi-channel microarrays.  The
methods used and form of measurements made for these two types differ, but the
research questions that are addressable using either type are the same.  In
both systems, DNA is immobilized on a fixed surface in a specific and
reproducible pattern.  This is the \emph{microarray}, sometimes abbreviated as
an \emph{array}.  The pattern that allows distinct regions on the
two-dimensional space of the surface to be mapped to the specific DNA sequences
immobilized at that address is known as a \emph{array design}.  Each distinct
region on the array is known as a \emph{feature} and the DNA sequence present
at a feature is called a \emph{probe}.  The source of the probe that is
immobilized at the feature can be a cDNA clone deposited onto the microarray,
or oligonucleotides that are synthesized on the microarray \emph{in-situ} using
photochemistry and light masks \cite{fodor,lipshutz}.

\subsection{Array Design and Fabrication}\label{ArrayDesign}

The first steps in performing a microarray experiment is deciding which genes
to assay on the microarray, how many genes to add, and how to arrange the genes
on the microarray.  These steps are collectively known as \emph{microarray
design}.

In earlier times it was commonplace to manufacture multi-channel microarrays in
the laboratory using a form of printer that deposited oligonucleotides onto a
glass slide.  This is no longer the case, as microarray production has been
industrialized and is now done on a scale and level of precision that requires
specialized apparata that are not practical to host in a laboratory
environment.  However, the design principles remain the same.  The array
designer needs to select the sequences that are to be assayed by the
microarray.  These can correspond toregions that are transcribed to RNA,
non-transcribed regions, or even oligonucleotides not expected to be observed
in the biological source material to be hybrized.  Historically, limitations in
manufacturing technology were very restrictive, and limited the number of genes
that could be assayed simultaneously.  Current technology is sufficiently
advanced to allow all genes to be simultaneously assayed for most organisms, so
the question of which and how many genes to add is no longer relevant.
However, as anyone who has performed and experiment will aver, the results
obtained from the same experimental procedure performed twice will yield
different results.  This also holds true for measurements made with
microarrays.  There are many potential sources of variance in the gene
measurements, and a good array design measures as much of this measurement
variance as possible.  Estimation of variance on a microarray essentially means
making multiple measurements of the same gene.  Each gene can be measured at
multiple distinct locations along its sequence.  Further, each location can be
measured multiple times.  A good array design uniformly distributes replicates
across the array surface to compensate for local perturbations on the array
surface itself.

Once the sequences to place onto the microarray have been determined, the
sequences are either placed onto the physcial substrate using printing
technology, or synthesized directly onto the substrate using photolithography
\cite{fodor,lipshutz}.  Fabrication and microarray design are interdependent,
in that improvements in manufacturing techniques allow for the fabrication of
more powerful devices.  Thus, this area of microarray technology is evolving
especially rapidly and any specific details of the synthesis I can describe
here will quickly be outdated.

Metadata are maintained that map the physical position of features in the
arraydesign of the microarray back to the sequences they were designed to match
during hybridization (Section \ref{Hybridization}).  These are used by the data
analyst after all data are collected to detect changes in gene expression that
correspond to changes in experimental factors.

\subsection{Hybridization}\label{Hybridization}

Once a microarray has been fabricated according to the array design, a solution
of DNA fragments, or \emph{targets}, can be labeled with a fluorescent dye and
\emph{hybridized}, or allowed to chemically anneal to the microarray's probes.
The property of reverse complementarity between subsets of the solvent and
immobilized DNA drives the solvent DNA targets to localize near to their
corresponding immobilized probes.

After hybridization, the amount of target DNA hybridized to each feature on the
array is assayed.  A laser is used to excite the fluorescent dye associated
with the targets, and a digital camera is used to measure the amount of
fluorescence present at each feature, and at the specific wavelength of
fluorescence known as a \emph{channel}.  The amount of fluorescent light is
assumed to be a monotonic and increasing function of the amount of target DNA
hybridized to that feature.  In the case of single-channel assays, only a
single fluorescent dye is used.  In a multi-channel assay, multiple target
solutions are hybridized to the array, each labeled with a different
fluorescent dye.  Each digital image of a microarray's fluorescence pattern for
a single channel is known as an \emph{image}, and the process of observing and
recording this image is called \emph{acquisition}.

For a single-channel assay, the acquired image is processed to produce a matrix
of numbers that has the same dimensions as the array design, and whose values
correspond to the absolute intensity of light present at each feature.  A
similar matrix is produced for a multi-channel assay as well, but because of
the competitive hybridization between the multiple labeled targets, the values
reported are not absolute, but rather the intensity of each channel relative to
all other target channels at a given feature.  Typically one of the channels
corresponds to a \emph{reference sample} that is a common denominator to
several arrays that will be analyzed as a set.  This allows the relative values
for the non-reference samples on each array to be compared.

For both single-channel and multi-channel arrays, the next step is
\emph{quantification}, a series of processes designed to convert the ``raw''
fluorescence intensities acquired from the scanner into values that are usable
in higher-level data analyses such as classification.  Methods for
quantification of microarray data are an area of intense interest
\cite{mas5,affy4,mbei,rma,vsn,gcrma,affyplm,seo,affybench} because they can
give very different estimates of the quantity of DNA present for a given
feature/channel, and thereby have a large effect on the biological conclusions
that can be drawn from analyses of those estimates.  Quantification methods are
discussed in greater detail in Section \ref{Quantification}.

%%%% XXX MERGE

\section{Microarray Assay Overview}
A successful microarray experiment is one which is able to provide insight into
the hypothesis for which it was designed.  Insights gained by experimentation
are hard-earned, and require careful planning and execution of multiple key
steps.

The initial steps of feature selection, microarray design, microarray
fabrication, experiment design, and biological sample processing, and image
acquisition are described generally in Sections
\ref{ArrayDesign}-\ref{Hybridization}.  A more exhaustive discussion of
protocol surrounding these steps is given in \cite{wit2004}.  Following their
acquisition, images are quantified, stored, distributed, analyzed, and
analytical results shared.  Challenges and findings related to these latter
procedures are the subject of \emph{\dbthesis}.  They are thoroughly discussed
in Sections \ref{Data Processing}-\ref{Informatics}.

The entire process, from microarray design to data analysis has been formally
described as an object model known as the Microarray Gene Expression Object
Model (MAGE-OM) \cite{mage}.  In this document I have borrowed vocabulary
defined in the MAGE-OM for clarity and consistency.  The MAGE-OM itself is
discussed in Section \ref{MAGE}.

\subsection{Microarray Design}
\label{ArrayDesign}

