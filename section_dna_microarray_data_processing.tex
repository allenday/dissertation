\subsection{Quantification}
\label{Quantification}

Microarray quantification, also known as \emph{pre-processing}, is the
process of estimating the quantity of each gene in the sample that was assayed
in the hybridization step (Section \ref{Hybridization}) of the experiment.
Quantification can be broken down into five distinct sub-procedures, executed
in the following order: image processing, background correction, normalization,
PM correction, and summarization.

\subsubsection{Image Processing}
\label{Image Processing}

\subsubsection{Background Correction}
\label{Background Correction}

Background correction is a statistical procedure that estimates and removes low
levels of noise on the microarray.  Background noise can have many sources.

The simplest and most common source of background noise is optical.  It can be
caused by general cross-hybridization of target to all probes, mis-calibration
of the microarray scanner's photo-sensor, and diffused or reflected light from
the laser used to excite the fluorescent dyes.  Optical noise can be estimated
by measuring the level of fluorescence from featureless regions of the
microarray and negative control probes that are not reverse-complementary to
any sequences in the hybridization mixture.  These measure background-level
reflected light and the level of non-specific hybridization, respectively.

Manufacturing and hybridization artifacts, such as surface scratches and salt
residues, are another source of noise.  A simple form of location-based
background correction is descibed in the Statistical Algorithms Description
Document \cite{affy:tech:2002}.   Briefly, the chip is broken into a $4x4$
grid of 16 rectangular regions.  The lowest 2\% of each region's probe
intensities are used to compute a background value for that region.  Each probe
(PM and MM) is then adjusted based upon a weighted average of the backgrounds
for all regions. The weights are based on the distances between the location of
the probe and the centroids of all regions.  More sophisticated methods attempt
to detect areas of the microarray containing high levels of manufacturing and
hybridization noise.  Noisy areas can be identified because the probes located
there will be outliers relative to probes for the same target located elsewhere
on the microarray.  Probes in these areas are considered unreliable and are
either given a very low weight parameter or are removed from normalization
(Section \ref{Normalization}) and other downstream processing (Sections \ref{PM
correction}-\ref{Summarization}) altogether \cite{affyplm}.

Newer, multi-array background correction methods have leveraged existing data
to build a models of how background noise is generally distributed.  The gcRMA
model \cite{gcrma} includes a parameter the sequence composition of each
probe, while other models such as those used in the RMA and MBEI
\cite{rma,bioc} methods only include a parameter for concordant each probe is
with other probes in the same set.  The RMA background correction method is the
\i{de facto} standard, and corrects perfect match (PM) probe intensities by
using a global model for the distribution of probe intensities. The model is
suggested by looking at plots of the empirical distribution of probe
intensities.  In particular the observed PM probes are modeled as the sum of a
normal noise component N (Normal with mean $\mu$ and variance $\sigma^2$) and a
exponential signal component S (exponential with mean $\alpha$). To avoid any
possibility of negatives, the normal is truncated at zero. Given we have O the
observed intensity, this then leads to an adjustment, given in Equation
\ref{rmabg}:

\begin{equation}
\label{rmabg}
E\left(s \lvert O=o\right) = a + b \frac{\phi\left(\frac{a}{b}\right) - \phi\left(\frac{o-a}{b}\right)}{\Phi\left(\frac{a}{b}\right) + \Phi\left(\frac{o-a}{b}\right) - 1 }
\end{equation}

where $a =  s- \mu - \sigma^2\alpha$ and $b = \sigma$. Note that $\phi$ and
$\Phi$ are the standard normal distribution density and distribution functions
respectively.  Note also that MM probe intensities are not corrected by either
of these routines \cite{rma,bioc}.

Multi-array background correction methods are able to detect background noise
due to the manufacturing and hybridization artifacts described above, but the
size of the aray artifact can be as small as a single feature.  This should in
principle do a better job of noise estimation.  A major drawback to multi-array
background noise models is that the noise estimates are only valid in the
context of the co-processed set of microarrays.  This is because the noise
estimates are derived from parameter estimates specific to that set of
microarrays.  While this is not a problem for small-scale analysis on
individual experiments, it creates difficulties when merging data from multiple
experiments because all microarrays will need to be re-processed to re-fit the
parameters of the noise model.  This re-processing problem can become
intractable for background correcting a very large number of arrays, and is
discussed in greater detail in Section \ref{Scalability}.

%%%%%%%%DEAD TEXT
%  Typical sources of probe-independent noise
%are the miscalibration of the photo-sensor used in Section \ref{Image
%Processing}, diffuse laser reflection from the microarray, debris and salt left
%over from hybridization (Section \ref{Hybridization}), and interaction between
%probes and non-target sequences.  Noise falls into two general categories:
%probe-independent and probe-specific.
%
%The level of probe-independent noise is estimated by measuring the level of
%fluorescence from featureless regions of the microarray and negative control
%probes that are not reverse-complementary to any sequences in the hybridization
%mixture.  These measure background-level reflected light and the level of
%non-specific hybridization, respectively.  Probe-independent background
%correction is generally done during image processing because the
%probe-independent noise tends to be spatially localized or uniformly
%distributed, although emerging methods include probe sequence in modeling
%background noise and correct each probe independently
%\cite{mbei,PMID_12582260}.  The level of probe-dependent noise is
%estimated by measuring the intensity of probes that differ only by a single
%nucleotide from their target sequence.  The idea to using these so-called
%\emph{mismatch probes} (MM) is that because their sequence is nearly identical
%to the sequence of the \emph{perfect match} (PM) probe that any difference
%between hybridization between a set of PM/MM pairs is almost entirely
%sequence-specific and thus provides a probe set-specific measure of noise
%\cite{rma,bioc}.

\subsubsection{Normalization}
\label{Normalization}

After correcting for background noise (Section \ref{Background Correction}),
microarrays are normalized.  The purpose of normalization is to transform the
distribution of microarray measurements so that properties of the
distribution of measurements match expectations.

The most simple form of microarray normalization is a linear scaling.  The
Affymetrix MAS 5 algorithm \cite{mas5} performs linear scaling by (1) setting
aside the top and bottom 1\% of measurements as outliers, adjusts the mean of
the remaining measurements to a constant value, then multiplies each
measurement, including the outliers, by the factor used to adjust the mean.

some properties of the normalized measurements fit an expected distribution.

\subsubsection{PM correction}
\label{PM correction}

\subsubsection{Summarization}
\label{Summarization}

The last step in microarray data preprocessing is to combine the measurements
from all probes in a probe set into a single value.  This procedure is called
\emph{summarization}.  The simplest summarization algorithm, called ``average
difference'' \cite{affy4} computes the mean of difference between each PM/MM
probe pair (Equation \ref{avgdiffsummary}),

\begin{equation}
\label{avgdiffsummary}
y_{k} = I_i^{-1}{\sum_{i=1}^{I_k} |PM_i-MM_i|}
\end{equation}

where the probe set $k$ has $PM$ perfect match and $MM$ mismatch probe pairs $i
= 1,\dots,I_k$.

Summarization methods parallel background correction and normalization methods
in that there are two varieties, the single-array methods and the multi-array
methods.  ``Average difference'' is an example of the former.  Multi-array
methods consider the distribution of probe measurements across all arrays, and
in some cases assign an array-specific parameter used to compute the probe set
summary.  The summarization component of the MBEI method introduced by Li and
Wong \cite{mbei,dchip} is given in Equation \ref{mbeisummary},

\begin{equation}
\label{mbeisummary}
y_{ij} = \phi_i \theta_j + \epsilon_{ij}
\end{equation}

where $y_{ij}$ is $PM_{ij}$ or the difference between $PM_{ij}-MM_{ij}$. The
$\phi_i$ parameter is a probe response parameter and $\theta_j$ is the
expression on microarray $j$.

The summarization component of RMA pre-processing \cite{rma} performss a
multi-array linear fit to data from each probe set.  Specifically, for probe
set $k$ with $i=1,\dots,I_k$ probe pairs and microarrays $j=1,\dots,J$ the
model given in Equation \ref{medianpolish} is fit,

\begin{equation}
\label{medianpolish}
\log_2\left(PM^{(k)}_{ij}\right) = \alpha_i^{(k)} + \beta_j^{(k)} + \epsilon_{ij}^{(k)}
\end{equation}

where $\alpha_i$ is a probe effect and $\beta_j$ is the $\log_2$ expression
value, and the method is known as \emph{median polish}, named after Tukey's
algorithm used to perform the calculation \cite{tukey:1977}.

It is noteworthy that summarized probe set values from all popular multi-array
summarization methods, including those described here, are dependent upon the
probe set and microarray effect parameters calculated as part of the model fit.
While this is not a problem theoritically, it introduces unique challenges in
the implementation of a pre-processing pipeline for a large number of arrays.
This is discussed in greater detail in Section \ref{Scalability}.

%\subsubsubsection{Normalization}
%%%%%%%%%%%%%%%%%%%%%%%%%%%%%%%%%%%%%%%%%%%%%%%%%%%%%%%%%<QUOTE>

You can see the background correction methods that are built into the package
by examining the variable \verb+bgcorrect.method+.

<<>>=
normalize.AffyBatch.methods
@
The Quantile, Contrast and Loess normalizations have been discussed and compared in \cite{PMID_12538238}.

\subsection{quantiles/quantiles.robust}

The quantile method was introduced by \cite{PMID_12538238}. The goal is to give each chip the same empirical distribution. To do this we use the following algorithm where $X$ is a matrix of probe intensities (probes by microarrays):

\begin{enumerate}
\item Given $n$ microarray of length $p$, form $X$  of dimension $p \times n$  where
each microarray is a column
\item Sort each column of $X$ to give $X_{\mbox{sort}}$
\item Take the means across rows of $X_{\mbox{sort}}$ and assign this mean to each element in the row to get $X'_{\mbox{sort}}$
\item Get $X_{\mbox{normalized}}$ by rearranging each column of $X'_{\mbox{sort}}$ to have the same ordering as original $X$
\end{enumerate}

The quantile normalization method is a specific case of the transformation $x'_{i} = F^{-1}\left(G\left(x_{i}\right)\right)$, where we estimate $G$ by the empirical distribution of each microarray and $F$ using the empirical distribution of the averaged sample quantiles.  Quantile normalization is pretty fast.

The \emph{quantiles} function performs the algorithm as above. The \emph{quantile.robust} function allows you to exclude or down-weight microarrays in the computation of $\hat G$ above. In most cases we have found that the \emph{quantiles} method is sufficient for use and \emph{quantiles.robust} not required.

\subsection{loess}

There is a discussion of this method in \cite{PMID_12538238}. It generalizes the $M$ vs $A$ methodology proposed in \cite{Dudoit:2002} to multiple microarrays. It works in a pairwise manner and is thus slow when used with a large number of microarrays.

\subsection{contrasts}

This method was proposed by \cite{astr:2003}. It is also a variation on the  $M$ vs $A$ methodology, but the normalization is done by transforming the data to a set of contrasts, then normalizing.

\subsection{constant}

A scaling normalization. This means that all the microarrays are scaled so that they have the same mean value. This would be typical of the approach taken by Affymetrix. However, the Affymetrix normalization is usually done after summarization (you can investigate \verb+affy.scalevalue.exprSet+ if you are interested) and this normalization is carried out before summarization.

\subsection{invariantset}

A normalization similar to that used in the dChip software \cite{mbei,dchip}. Using a baseline microarray, microarrays are normalized by selecting invariant sets of genes (or probes) then using them to fit a non-linear relationship between the ``treatment'' and ``baseline'' microarrays. The non-linear relationship is used to carry out the normalization.

\subsection{qspline}
This method is documented in \cite{workman:etal:2002}. Using a target microarray (either one of the microarrays or a synthetic target), microarrays are normalized by fitting splines to the quantiles, then using the splines to perform the normalization.
%%%%%%%%%%%%%%%%%%%%%%%%%%%%%%%%%%%%%%%%%%%%%%%%%%%%%%%%%</QUOTE>

%\subsubsubsection{PM Correct Methods}
%%%%%%%%%%%%%%%%%%%%%%%%%%%%%%%%%%%%%%%%%%%%%%%%%%%%%%%%%<QUOTE>
<<>>=
pmcorrect.methods
@
\subsection{mas}

An \emph{ideal mismatch} is subtracted from PM. The ideal mismatch is documented by \cite{affy:tech:2002}. It has been designed so that you subtract MM when possible (i.e. MM is less than PM) or something else when it is not possible. The Ideal Mismatch will always be less than the corresponding PM and thus we can safely subtract it without risk of negative values.

\subsection{pmonly}

Make no adjustment to the pm values.

\subsection{subtractmm}

Subtract MM from PM. This would be the approach taken in MAS 4 \cite{affy4}. It could also be used in conjunction with the Li-Wong model.
%%%%%%%%%%%%%%%%%%%%%%%%%%%%%%%%%%%%%%%%%%%%%%%%%%%%%%%%%</QUOTE>

vsn			\cite{vsn}
genelogic		\cite{genelogic}

A few sentences to dismiss 2-channel microarrays.  There is a parallel set of conceptually identical problems, but the details for how to solve them differ.
Need data consistency.  How to get it
