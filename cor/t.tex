% <execute lang="R">
% #now build up the "repeat offender" vector
% source('/home/allenday/cvsroot/celsius/dump/lib/R/Celsius/IO/exs.R');
% conn = Celsius.exs('/home/allenday/cvsroot/celsius/dump/data2/HG-U133_Plus_2.rma_trim1');
% ctrl1=e_names(conn)[substr(e_names(conn), 1, 4)=="AFFX"]
% mat = get(conn, ctrl1, s_names(conn)); dim(mat); mat=t(mat); dim(mat);
% ### mat now has 12100 rows and 62 columns.
% J = length(ctrl1); # J is actually ncol(mat)
%
% #### The following considers 1 cutoff value ####
% residual.cutoff = 3;
% residual.tally  = rep(0, nrow(mat) ); r2.tally=rep(NA, J);
%
% for ( j in 1:J ) {
%   print(j);
%   mat.lm = lm(mat[,j] ~ mat[,-j]);
%   #the mean residual is always 0.
%   #mat.residual.sd = as.vector( ( residuals(mat.lm) - mean(residuals(mat.lm)) ) / sd(residuals(mat.lm)) );
%   mat.residual.sd = residuals(mat.lm) / sd(residuals(mat.lm));
%   residual.tally = residual.tally + (abs( mat.residual.sd ) > residual.cutoff);
%   r2.tally[j] = summary(mat.lm)$r.squared
% }
% #rownames(residual.tally) = rownames(mat);
% table(residual.tally)
% sum(residual.tally > J*0.1 )
% # residual.tally = residual.tally / J;
% # rt3=residual.tally
% plot(sort(residual.tally)); dev.off()
% #summary(r2.tally)
% postscript('probeset_residuals.ps',width=8,height=8);
% plot(sort(residual.tally), xlab='arrays, sorted by r^2', ylab='outlier frequency', main='array outlier frequency');
% abline(0.05, 0, col='red');
% dev.off()
% r2.tally
% write.table(ctrl1,'control_probesets.tex', row.names=FALSE, col.names=FALSE );
% write.table(residual.tally, "residual.tally.txt", sep="\t")
% write.table( (1:12100)[residual.tally>5] , 'outlier.badrow', row.names=FALSE, col.names=FALSE );
% </execute>

