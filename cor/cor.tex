\documentclass{article}
\usepackage{graphicx}
\usepackage{times}
\usepackage{setspace}
\usepackage{url}
%\usepackage[final]{pdfpages}
%\usepackage{verbatim}
%\usepackage{chapterbib}
%\usepackage{ifthen}
%\usepackage{longtable,lscape}
\begin{document}
\title{The Title}
\author{
Allen Day \and Stanley F. Nelson
% Allen Day\correspondingauthor$^{1,2,\ddag}$ \email{Allen Day\correspondingauthor - allenday@ucla.edu} \and
% Brian D. O'Connor$^{2,3,\ddag}$ \email{Brian D. O'Connor - boconnor@ucla.edu} \and
% Jordan Mendler$^{1}$ \email{Jordan Mendler - jmendler@ucla.edu} \and
% Jared Fox$^{1,4}$ \email{Jared Fox - jaredfox@ucla.edu} \and
% Stanley F. Nelson$^{1}$ \email{Stanley F. Nelson - snelson@ucla.edu} \and
% Lincoln D. Stein\correspondingauthor$^{2}$  \email{Lincoln D. Stein\correspondingauthor - lstein@cshl.edu}
}
\maketitle

\section{Abstract}\label{Abstract}
...

\section{Introduction}\label{Hypothesis}
Genes can be annotated by observing their expression pattern across a large
number of anonymous, unlabeled samples.  Further, we assert that
non-orthologous genes can be annotated cross-species given 1) gene expression
observations in two species, and 2) a partial, sequence-based ortholog map
between the two species.

\section{Methods}\label{Methods}

\subsection{Data Assembly}\label{Assembly}
We use the gene expression data assembled as part of the Celsius project
(\url{http://genome.ucla.edu/projects/celsius}) \cite{XXX}.  Specifically, we
assemble into two separate matrices all observations made on current-generation
Affymetrix GeneChip ArrayDesigns for human and mouse.  For the human
HG-U133\_Plus\_2 ArrayDesign, our matrix size is 12827 microarrays by 54675
probesets.  For the mouse Mouse430\_2 ArrayDesign, our matrix size is 7345
microarrays by 45101 probesets.

\subsection{Data Processing}\label{Processing}
%http://www.nature.com///msb/journal/v3/n1/full/msb4100149.html
The number of human probesets analyzed was 27337, and the number of mouse
probesets analyzed was 22550.  For each ArrayDesign, the number of probesets is
half of the number present on each ArrayDesign.  The probesets used were
selected by identifying those with variance greater than the median probeset
performance for the ArrayDesign.  Komurov \& White demonstrated that expression
variance is positively correlated with the magnitude of modulation of gene
expression across a wide variety of conditions and does not selectively enrich
for low-abundance genes \cite{Komurov}.  A variance measure reflects the extent
of gene regulation rather than measuring artifacts owing to low mRNA abundance.

%http://www.kamalnigam.com/papers/canopy-kdd00.pdf
For each ArrayDesign, a correlation matrix was created from each of the
variance-constrained matrices.  Each probeset was then assigned to one or more
canopies using the canopy-clustering algorithm \cite{McCallum} using the
correlation measurements to all other probesets.  We used $(k_i)=0.1$
correlation coefficient units as the inner distance cutoff, and i$(k_o)=0.2$
units as the outer distance cutoff.  The value of $k_i$ was selected as the
minimum distance that prevented $>90\%$ of probesets for the same gene from being
split into multiple inner canopies, and a plot of ortholog probeset
fragmentation as a function of varying $k_i$ is given in Figure
\ref{figure:k.i}.  The value of $k_o$ was selected as the maximum distance that
prevented $>90\%$ of probesets from appearing in more than one outer canopy, and
a plot of probeset multiple assignment as a function of varying $k_o$ is given
in Figure \ref{figure:k.o}.  The result of clustering using these values of
$k_i$ and $k_o$ yielded XXX human centroids and XXX mouse centroids.

\subsection{Ortholog Criteria}\label{Orthologs}
%http://www.pubmedcentral.nih.gov/articlerender.fcgi?artid=165568

%%%count unique gene names that have a mouse ortholog
%cat HG-U133_Plus_2.na24.ortholog.csv | grep -F -f half/HG-U133_Plus_2.e.cor.probeset | grep Mouse430_2 | awk -F '","' '{print $5}' | sort | uniq | wc

%%%count unique human probesets that have a mouse ortholog
%cat HG-U133_Plus_2.na24.ortholog.csv | grep -F -f half/HG-U133_Plus_2.e.cor.probeset | grep Mouse430_2 | perl -ne 'm#^"(.+?)"#;print "$1\n"' | sort | uniq | wc

%%%count unique mouse probesets that have a human ortholog
%cat HG-U133_Plus_2.na24.ortholog.csv | grep -F -f half/HG-U133_Plus_2.e.cor.probeset | grep Mouse430_2 | perl -ne 'm#.*"(.+?)","Mouse430_2#;print "$1\n"' | sort | uniq | wc

%%%count the number of unique human probesets per gene name
%cat HG-U133_Plus_2.na24.ortholog.csv | grep -F -f half/HG-U133_Plus_2.e.cor.probeset | grep Mouse430_2 | awk -F '","' '{print $5,"\t",$1}' | sort | uniq | awk -F '     ' '{print $1}' | sort | uniq -c | sort -rn | less

%%%count the number of unique mouse probesets per gene name
%cat HG-U133_Plus_2.na24.ortholog.csv | grep -F -f half/HG-U133_Plus_2.e.cor.probeset | grep Mouse430_2 | awk -F '","' '{print $5,"\t",$3}' | sort | uniq | awk -F '     ' '{print $1}' | sort | uniq -c | sort -rn | less

%%%for mouse probesets per gene name distribution figure
%cat HG-U133_Plus_2.na24.ortholog.csv | grep -F -f half/HG-U133_Plus_2.e.cor.probeset | grep Mouse430_2 | awk -F '","' '{print $5,"\t",$3}' | sort | uniq | awk -F '     ' '{print $1}' | sort | uniq -c | awk '{print $1}' | sort -n | uniq -c

%%%for human probesets per gene name distribution figure
%cat HG-U133_Plus_2.na24.ortholog.csv | grep -F -f half/HG-U133_Plus_2.e.cor.probeset | grep Mouse430_2 | awk -F '","' '{print $5,"\t",$1}' | sort | uniq | awk -F '     ' '{print $1}' | sort | uniq -c | awk '{print $1}' | sort -n | uniq -c


We used the ``na24'' build of the Affymetrix NetAffx probeset annotation
\cite{Liu} for the HG-U133\_Plus\_2 ArrayDesign as our list of human/mouse
orthologs.  This annotation set identifies 11357 human/mouse ortholog pairs,
and each gene in the pair is represented by one or more probesets on the
corresponding ArrayDesign.  After constraining for high-variance probesets, we
are left with a working set of 17401 human probesets and 22364 mouse probesets.
The distribution of probesets per ortholog, partitioned by species, is given in
Figure \ref{figure:hmp}.


  There is not a 1:1 correspondence between
human and mouse probesets for two reasons.  The first reason is that on a given
ArrayDesign, a gene transcript may be assayed by more than one probeset.
Sometimes these probesets are specific to a single gene transcript, and other
times they cross-hybridize to many gene transcripts.  Thus, there is a
many-to-many relationship between the probesets targeting human/mouse
orthologs.

\subsection{Cluster alignment}\label{Alignment}

In cases where all probesets that represent a single human gene are found in a
single human probeset cluster $c_H$, and all mouse probesets corresponding to
the orthologous mouse gene are found in cluster $c_M$.

Because of the one-to-many relationship between genes and probesets for both
the human and mouse ArrayDesigns, and because of the many-to-many relationship
between probesets for human/mouse orthologs (Section \ref{Orthologs}), it
sometimes occurs that a single human gene is represented by multiple probeset
clusters $c{\in}C$, and also that each human probeset cluster may correspond to
multiple mouse probeset clusters (Section \ref{Processing}).

In these cases, identification of orthologous clusters of expression between
human and mosue requires calculating the probability that a human probeset
cluster $c_H$ corresponds to a mouse cluster $c_M$.  We regard the
sequence-based orthologs as authoritative, and use these as a ``scaffold'' for
aligning probeset clusters using Equations \ref{equation:s}, \ref{equation:mc},
and \ref{equation:m}.  The probability of a single probeset occuring in one of
all clusters $c{\in}C$ is denoted $s(c)$, and is defined as:

\begin{equation}
\label{equation:s}
s(c)={size\_of(c)}\over{\sum_{g}^{C}s(g)}
\end{equation}

Thus, the probability of any subset of all probesets $m{\in}M$ being found in
the same cluster $c$ is:

\begin{equation}
\label{equation:mc}
p(m{\in}c)=\prod_{1}^{M}s(c) 
\end{equation}

and more generally, the probability of any subset of all probesets $m{\in}M$
being found in any cluster $c{\in}C$ is:

\begin{equation}
\label{equation:m}
p(m)=\prod^{m}\sum_{c=1}^{C}{{s(c)}\over{C}}
\end{equation}

We iterated across all human genes, and for each identified the human probeset
cluster that contained the largest number of probesets representing that gene.
This produced a many-to-one mapping between human genes and probeset clusters.
Then, we calculated $p(m)$ for each group of mouse probesets that represented
the mouse ortholog and were assigned to different mouse clusters.  The best
correspondence between a human probeset cluster and a mouse probeset cluster
was identified by minimizing $p(m)$.


----------\linebreak

\newpage
\section{Figures}\label{Figures}

\begin{figure}[h]
\label{figure:k.i}
\caption{Caption for Figure ``k.i''}
\end{figure}
\pagebreak

\begin{figure}[h]
\label{figure:k.o}
\caption{Caption for Figure ``k.o''}
\end{figure}
\pagebreak

\begin{figure}[h]
\label{figure:hmp}
\includegraphics[angle=-90,width=12cm]{hmp.ps}
\caption{Number of probesets assigned to a single orthologous gene.}
\end{figure}
\pagebreak

\end{document}
