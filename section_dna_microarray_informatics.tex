\subsubsection{Data Modeling}
\label{Data Modeling}

In order to be able to perform analyses on the results of a DNA microarray
experiment, the assay data collected must be structured and stored.  This is
also true of the experimental \emph{metadata}, or description of the
experimental design and procedures.  Principles of scalability and
mass-production prescribe that the encoding of these data be uniform, meaning
that the structure used to encode the experiment must be sufficiently flexible
and descriptive to capture the full range of experiments that can be conceived.

The \emph{MicroArray Gene Expression} (\emph{MAGE}) Group was established by
the \emph{MicroArray Gene Expression Data Society} (\emph{MGEDS}, sometimes
also referred to as \emph{MGED}) to develop a standard for the representation
of microarray data, with the intent of making those data exchangeable between
different information systems.  The work of the MAGE Group can be divided into
two types of projets: those dealing with the syntax used to concretely
represent microarray data, and those dealing with the symantics used to
describe the microarray experimental materials and processes, produced data,
and data derived from processing.

The semantic developments of the MAGE Group are more important for \dbthesis
than the syntactic developments.  This is because semantics are abstract and
generally useful for encoding experimental information, while the syntactic
developments of the MAGE Group have optimized for the purposes of data
interchange using technologies such as the \emph{eXtensible Markup Language}
(\emph{XML}) that are not well-suited for systems whose primary purpose is to
store and retrieve large volumes of quantitative data.

Application of the MGED semantic technologies are discussed in other sections
of this document.  There are two major semantic developments from MGEDS.  The
first of these is the \emph{MAGE Object Model}, or \emph{MAGE-OM}.  The purpose
of the MAGE-OM is to provide a standard set of classes that can be used to
represent any object or process that may be included in, or referenced from, a
microarray experiment.  The model employs concepts common to object-oriented
software design and knowledge engineering, such as subclass/superclass
relationships between object classes, the notion of abstract classes, and the
possibility for directional and cardinal relationships between objects.  The
specification of the MAGE-OM was developed using the \emph{Unified Modeling
Language} (\emph{UML}), a standard technology used by knowledge and software
engineers for the composition of object models.  Fitting microarray experiment
data into the MAGE Object Model is described in greater detail in Section
\ref{Experimental Metadata}.  Encoding specific information about objects under
investigation or that are used to facilitate the conduct of an experiment are
discussed in Section \ref{Object Metadata}.

Unique syntax-related challenges that have arisen as part of \dbthesis are
also presented.  Internal data representation and scalability issues are
discussed in Section \ref{} and interoperability concerns are discussed in
greater detail in Section \ref{}.

\subsubsection{Experimental Metadata}
\label{Experimental Metadata}

As a concrete example of how an experiment might be encoded into a MAGE-ML
document, consider the now-classical study of leukemias by Golub, et al
\cite{golub}.  Briefly, this study is has both pattern-detection and
predictive-modeling methodology (Sections \ref{Pattern
Detection}-\ref{Predictive Modeling}), and describes a method for identifying
features that discriminate between two classes of leukemia and how they can be
used to identify the class of previously unlabeled cancer samples.

Encoded into the MAGE-OM, the initial cancer samples in Golub, et al
\cite{golub} are represented as \emph{BioSource} objects, a class used for
biological material prior to any treatment.  Each BioSource object then goas
through a series of modifications, ultimately resulting in a
\emph{LabeledExtract} object that represents the fluorescently-labeled cRNA
that is hybridized onto a microarray.  In the series of modifications, a
combination of a \emph{BioSample} object and one or more \emph{Treatment}
objects is used to represent the transformed Biosource object.  Further, each
LabeledExtract refers back to the object from which it was derived all the way
back to the BioSource object so that the full path of derivation via treatment
is modeled.

To represent the microarray hybridization, a \emph{BioAssay} object is created.
The BioAssay is a central connection point in the object model.  It refers to
an \emph{Array} object representing the microarray itself, the LabeledExtract
that is hybridized, and to one or more \emph{Factor} objects.  The factor
objects are in turn related to a network of other objects that encode the
design and variables used in the microarray experiment.  The Array object is
associated to a series of other objects that describe the microarray itself,
including the information about specific sequences and their physical locations
on the microarray, as well as information about the grouping of features on the
array used as reporters for a common cRNA target sequence.  Further, each
LabeledExtract is associated with a \emph{Channel} object.  The Channel is used
to link a specific LabeledExtract to a specific \emph{Image} object that
results from the scanning of the hybridized array.

The Image object is combined with a \emph{FeatureExtraction} object to produce
a \emph{BioAssayData} object.  This association of objects represents that
transformation of the microarray image acquired by the scanner (Section
\ref{Assay}) into a numerical form that can be processed (Section \ref{Data
Processing}) for further analysis.  BioAssayData objects may also be derived
from other BioAssayData objects, similar to the way BioSource, BioSample, and
LabeledExtract objects may be related.  This is how the MAGE-OM represents
arbitrary data transformations, and is sufficient for describing microarray
data pre-processing (Section \ref{{Data Processing}), as well as additional
downstream summarizations or other transformations of these data.

\subsubsection{Object Metadata}
\label{Object Metadata}

Objects in the MAGE-OM may have attributes attached to them to provide
more specific detail about the microarray experiment.  A design decision was
made to reference objects external to the MAGE-OM and whose primary purpose is
description because doing so constrains the scope of MAGE-OM to describing only
the structure of the microarray experiment and associated quantitative data.
Object metadata is discussed in greater detail in Section \ref{Object
Metadata}.

%XXX

The second area of development in microarray semantics is the \emph{MGED
Ontology}, or \emph{MO}.  The purpose of MO is to allow MAGE-OM objects to be
described in greater detail.  A trade-off was made to use an \emph{ontology},
or a structured controlled vocabulary, in association with the MAGE-OM to keep
the scope of the MAGE-OM itself focused on the description of experimental
processes, 
%XXX

\subsection{Data Storage & Retrieval}
\label{Data Storage & Retrieval}

The data and metadata produced as part of pre-processing (Section \ref{Data
Processing}) and fitting the microarray experiment to a uniform data model
(Section \ref{Data Modeling}) must be stored and made available to analysts for
further processing

\subsection{Interoperability & Exchange Protocols}
\label{Protocol}

